% Energy Policy cover letter template
\documentclass[11pt]{letter}
\usepackage{hyperref}
\signature{Jinsu Park\\PLANiT Institute}
\address{PLANiT Institute\\Seoul, Republic of Korea}
\begin{document}
\begin{letter}{Editor-in-Chief\\Energy Policy}
\opening{Dear Editor,}

Please find enclosed the manuscript titled ``Testing the limits of carbon pricing: Can Korea's emissions trading system align steel industry investments with national climate targets?'' for consideration in \emph{Energy Policy}. The paper provides the first optimisation-based assessment of Korea's K-ETS price trajectory against a firm-level carbon budget, demonstrating that current policy settings still overshoot POSCO's allocation by 4\% even under the NGFS Net Zero pathway.

\textbf{Key contributions}
\begin{itemize}
  \item Mixed-integer optimisation of POSCO's technology portfolio under NGFS carbon price scenarios, capturing investment lumpiness and scrap constraints.
  \item Quantification of the policy-performance gap using a sectoral carbon budget derived from Korea's NDC and net-zero pledge.
  \item Policy package showing how accelerated allowance phase-out, price floors, and scrap infrastructure can close the remaining budget overshoot while reducing total system costs.
\end{itemize}

The manuscript has not been published previously and is not under consideration elsewhere. All co-authors have approved the submission and agree with its content.

Suggested reviewers:
\begin{enumerate}
  \item Prof. Oliver Sartor (Columbia Center on Global Energy Policy) --- expertise in industrial carbon pricing.
  \item Dr. Valentin Vogl (Lund University) --- specialist in steel decarbonisation pathways.
  \item Dr. Meredith Fowlie (University of California, Berkeley) --- expert in emissions trading systems and industrial policy.
\end{enumerate}

Thank you for considering this submission. I would be pleased to provide any additional information required.

\closing{Sincerely,}

\end{letter}
\end{document}

