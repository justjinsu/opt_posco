\documentclass[preprint,1p,authoryear]{elsarticle}

% ===== Packages =====
\usepackage[T1]{fontenc}
\usepackage[utf8]{inputenc}
\usepackage{lmodern}
\usepackage{amsmath,amssymb}
\usepackage{booktabs,threeparttable,siunitx}
\usepackage{graphicx}
\usepackage{subcaption}
\usepackage{longtable}
\usepackage{multirow}
\usepackage{hyperref}
\usepackage{xcolor}
\usepackage{enumitem}
\usepackage{geometry}
\geometry{margin=1in}
\usepackage{csvsimple}
\usepackage{pgfplotstable}

% Graphics path
\graphicspath{{./}}
\journal{Energy Policy}
\biboptions{authoryear}


% ===== Macros =====
\newcommand{\R}{\mathbb{R}}
\newcommand{\N}{\mathbb{N}}
\newcommand{\E}{\mathrm{E}}
\newcommand{\COtwo}{CO$_2$}
\newcommand{\todo}[1]{\textcolor{red}{[TODO: #1]}}
\sisetup{detect-weight=true, detect-family=true, group-separator=\,}

% ===== Title & Authors =====
\begin{document}
\begin{frontmatter}
\title{Carbon Pricing and Industrial Decarbonization: Can Korea's ETS Drive Low-Carbon Investment in Steel?}

\author[planit]{Jinsu Park}
\address[planit]{PLANiT Institute, Seoul, Republic of Korea}

% ===== Highlights (Energy Policy requires 3--5 bullet highlights) 

\section*{Highlights}
\begin{itemize}[leftmargin=*]
  \item \textbf{Hydrogen pathway viability}: Under Net Zero carbon pricing (\$383/tCO$_2$ by 2030, \$638/tCO$_2$ by 2050), hydrogen-based DRI captures 41\% of POSCO's 2050 output with cumulative emissions of 1,169~MtCO$_2$—overshooting the sectoral budget by only 5.3\%, validating POSCO's hydrogen strategy.
  \item \textbf{Binary pricing outcome}: Ambitious Net Zero pricing enables hydrogen deployment and near-budget alignment, while moderate scenarios (Below~2$^\circ$C, NDCs) fail catastrophically with +78\% overshoots, demonstrating that half-measures cannot justify transformative hydrogen investment.
  \item \textbf{Infrastructure dependency}: Hydrogen pathway requires dedicated supply chains (production, transport, pellet feedstock) but proves cost-competitive at \$815/t versus \$861/t under business-as-usual, showing early capital investment substitutes for recurring carbon costs.
  \item \textbf{Price thresholds matter}: Hydrogen DRI becomes economic when carbon prices exceed \$350--400/tCO$_2$, explaining why only Net Zero pricing justifies POSCO's hydrogen investment while weaker scenarios lock in conventional blast furnaces.
  \item \textbf{Policy enablers}: Realising the hydrogen pathway requires sustained price floors above \$600/tCO$_2$ by 2050, faster free-allocation phase-out, government co-investment in H$_2$ infrastructure, and long-term supply contracts that de-risk hydrogen steelmaking.
\end{itemize}

% ===== Abstract =====
\begin{abstract}
Can carbon pricing justify Korea's ambitious shift toward hydrogen steelmaking? We examine this question using a mixed-integer optimisation model that evaluates POSCO's technology choices under NGFS Phase V carbon-price trajectories (Net Zero 2050: \$383/tCO$_2$ in 2030, \$638/tCO$_2$ in 2050; Below~2$^\circ$C: \$71/\$166; NDCs: \$118/\$130) against a 1,110~MtCO$_2$ sectoral budget for 2025--2050. We find that ambitious Net Zero pricing enables hydrogen-based direct reduction to capture 41\% of 2050 output, achieving cumulative emissions of 1,169~MtCO$_2$—just 5.3\% above budget. This validates POSCO's hydrogen strategy: the optimisation independently selects H$_2$-DRI as the primary decarbonisation route when carbon prices sustain above \$350--400/tCO$_2$. Moderate pricing scenarios fail catastrophically—Below~2$^\circ$C and NDCs both overshoot by 78\%—because prices never justify the upfront hydrogen investment, locking in conventional blast furnaces through 2050. The hydrogen pathway proves cost-competitive at \$815/t steel versus \$861/t under business-as-usual, as early capital substitutes for recurring carbon costs. However, realising this pathway requires more than pricing alone: government must co-invest in hydrogen infrastructure (production, transport, pellet feedstock), accelerate free-allocation phase-out so firms face real carbon costs, and provide long-term supply contracts that de-risk the transition. Our results show carbon pricing can work—but only when sustained at levels that justify transformative investment in hydrogen technology.
\end{abstract}

\begin{keyword}
Hydrogen steelmaking \sep carbon pricing \sep steel decarbonization \sep mixed-integer optimization \sep Korea ETS \sep NGFS scenarios \sep industrial policy
\JEL{Q41, Q54, L61}
\end{keyword}
\end{frontmatter}

% ===== 1. Introduction =====
\section{Introduction}

Can carbon pricing justify Korea's ambitious shift toward hydrogen steelmaking? POSCO—the world's sixth-largest steel producer—has committed to hydrogen-based direct reduction as its primary decarbonization pathway, with its HyREX demonstration plant operational since 2021. This strategic bet requires wholesale infrastructure replacement: individual hydrogen DRI plants cost over \$2 billion and operate for 25-40 years, creating irreversible technology lock-in \citep{MaterialEconomics2019}. Whether Korea's emissions trading system can make such investments profitable remains untested.

The stakes are considerable. POSCO's emissions represent 10\% of Korea's greenhouse gas inventory, roughly equivalent to Belgium's entire carbon footprint \citep{KOSIS2023}. Korea's ETS, launched in 2015, now covers 70\% of national emissions but historically shielded steel through 95\% free allocation \citep{kim2021kets, ICAP2024}. This protection is scheduled to decline as Korea pursues 2030 NDC targets (40\% reduction) and 2050 carbon neutrality, gradually exposing producers to meaningful price signals.

We test whether carbon pricing can align POSCO's investments with Korea's sectoral carbon budget of 1,110 MtCO$_2$ for 2025-2050, derived from national climate commitments and POSCO's 60\% domestic market share \citep{korea2020carbon}. Using mixed-integer optimization under three NGFS Phase V carbon price scenarios—Net Zero 2050 (\$383/tCO$_2$ by 2030, \$638 by 2050), Below 2$^\circ$C (\$71/\$166), and NDCs (\$118/\$130)—we model POSCO's technology portfolio choices \citep{NGFS2024}. Unlike prior studies treating capacity as continuous variables, we explicitly model discrete blast furnace investment cycles.

Our findings validate POSCO's hydrogen strategy. Under Net Zero pricing, hydrogen-based DRI captures 41\% of 2050 output, achieving cumulative emissions of 1,169 MtCO$_2$—just 5.3\% above budget. This validates hydrogen as the cost-optimal primary pathway when carbon prices sustain above \$350-400/tCO$_2$. However, moderate pricing scenarios fail catastrophically: Below 2$^\circ$C and NDCs both overshoot by 78\%, never justifying upfront hydrogen investment and locking in conventional blast furnaces through 2050. The hydrogen pathway proves cost-competitive (\$815/t versus \$861/t under business-as-usual) through capital substitution—early investment avoids recurring carbon costs. Yet realizing this pathway requires more than pricing alone: government must co-invest in hydrogen infrastructure, accelerate free-allocation phase-out, and provide long-term contracts that de-risk the transition.

\section{Literature Review}

Steel decarbonization research identifies three main technology routes: carbon capture retrofitted to blast furnaces (CCUS), scrap-based electric arc furnaces (EAF), and hydrogen-based direct reduction (H$_2$-DRI) \citep{IEA2020steel}. Early studies focused on levelized cost comparisons \citep{Vogl2018, Otto2017}, consistently finding hydrogen uneconomic absent carbon prices exceeding \$150/tCO$_2$. Recent work has grown more sophisticated: \citet{prammer2021steel} examined H$_2$-DRI learning curves projecting 20-30\% cost reductions by 2040, while \citet{ueckerdt2021potential} explored hydrogen infrastructure co-evolution across sectors. System constraints have received growing attention—\citet{pauliuk2013global} documented scrap availability limits constraining EAF expansion to 40-45\% of production, and \citet{wang2021hydrogen} mapped regional hydrogen production potential finding costs of \$3-4/kg in Asia versus <\$2/kg in Northern Europe by 2040.

Three gaps persist. First, optimization models treat technology adoption as continuous variables, missing the lumpy nature of steel investment—blast furnaces are discrete 2-4 Mt/year units with 40-year lifespans creating path dependencies \citep{Griffin2020}. Second, few integrate region-specific constraints: Korea's tight scrap availability, limited renewables, and industrial concentration differ fundamentally from European contexts \citep{zhang2022steel}. Third, no research tests whether actual carbon price trajectories can drive transitions at scales needed for sectoral carbon budget compliance.

Carbon pricing effectiveness literature shows mixed results. Power generation demonstrates clear price sensitivity \citep{jarke2017carbon}, but manufacturing evidence is limited. \citet{martin2016industry} documented emissions leakage in EU ETS, while \citet{sartor2012benchmark} showed Phase III free allocation removed carbon cost exposure for steel. \citet{demailly2018european} estimated prices exceeding €50/tCO$_2$ needed for H$_2$-DRI competitiveness. Korea's K-ETS shows limited industrial restructuring despite rising prices \citep{kim2021kets}.

Carbon budget literature established that cumulative emissions determine temperature outcomes \citep{matthews2009proportionality}, but sectoral allocation remains contentious. \citet{kuramochi2018beyond} examined decomposition approaches without clear allocation principles. \citet{bataille2018role} estimated steel could consume 15-20\% of global budgets under business-as-usual, yet \citet{Griffin2020} noted we lack frameworks for testing policy adequacy against sectoral budgets.

This paper addresses these gaps by integrating technology optimization with carbon budget evaluation. We test whether NGFS carbon price scenarios can drive H$_2$-DRI deployment consistent with Korea's sectoral budget of 1,110 MtCO$_2$ (2025-2050), using mixed-integer formulation capturing discrete blast furnace cycles. Unlike prior studies using arbitrary targets, we benchmark against budgets derived systematically from Korea's NDC and 2050 neutrality commitments.

% ===== 2. Methods: Optimization model =====
\section{Methodology}

We develop a mixed-integer linear programming (MILP) model that determines POSCO's least-cost technology portfolio over 2025-2050, explicitly capturing lumpy investment decisions (discrete blast furnace units, not fractional capacity), long asset lifetimes (25-40 years), and operational constraints. The model minimizes discounted total system costs—capital expenditure, fixed/variable operating costs, and ETS compliance payments—subject to demand satisfaction, capacity evolution, emission accounting, and technology readiness constraints.

\subsection{Model formulation}

\textbf{Decision variables}: $B_{r,t} \in \mathbb{Z}_{\ge 0}$ (integer units of route $r$ built in year $t$), $K_{r,t} \in \mathbb{R}_{\ge 0}$ (available capacity), $Q_{r,t} \in \mathbb{R}_{\ge 0}$ (annual production), and $ETS_{t}^{+} \in \mathbb{R}_{\ge 0}$ (net carbon liabilities).

\textbf{Objective}: Minimize net present value of costs:
\begin{align}
\min \sum_{t \in \mathcal{T}} \delta_t \left[ \sum_{r} B_{r,t} \kappa_r c^{capex}_r + \sum_{r} K_{r,t} c^{fixom}_r + \sum_{r} Q_{r,t} \left(\sum_i \alpha_{r,i} p_{i,t}\right) + P^{CO_2}_t \cdot ETS_t^+ \right] \cdot 10^6,
\end{align}
where $\delta_t = (1+\rho)^{-(t-2025)}$ with $\rho=0.05$ discount rate.

\textbf{Constraints}: Demand satisfaction $\sum_r Q_{r,t} = D_t$; capacity utilization $Q_{r,t} \le 0.9 K_{r,t}$; capacity evolution $K_{r,t} = K_{r,t-1} + \kappa_r B_{r,t} - R_{r,t}$ with retirements $R_{r,t}$ following 40-year (BF) or 30-year (EAF/DRI) lifetimes; ETS liability $ETS_t^+ \ge \sum_r ef_r^{net} Q_{r,t} - A_t^{free}$ with net factors $ef_r^{net} = ef_r^{gross}(1-\eta^{CCUS})$ for CCUS routes ($\eta^{CCUS}=0.80$); technology readiness $B_{r,t}=0$ for $r \in \mathcal{R}^{H_2}$ when $t<2030$ and $r \in \mathcal{R}^{CCUS}$ when $t<2027$; and integer investment $B_{r,t} \in \mathbb{Z}_{\ge 0}$.

\subsection{Carbon budget and policy-performance gaps}

Korea's commitments (40\% reduction by 2030, carbon neutrality by 2050) yield a sectoral steel budget via proportional allocation: $CB^{POSCO} = \sum_{t=2025}^{2050} E^{national}_t \cdot \phi^{steel} \cdot \phi^{POSCO} \approx 1{,}110$ MtCO$_2$, where $\phi^{steel}=0.12$ (steel's emissions share) and $\phi^{POSCO}=0.6$ (POSCO's market share). The policy-performance gap quantifies budget compliance:
\begin{align}
\text{Gap}_s = \frac{\sum_t E_{s,t}^{optimal} - CB^{POSCO}}{CB^{POSCO}} \times 100\%, \quad E_{s,t}^{optimal} = \sum_r ef_r^{net} Q_{r,t}^*(s).
\end{align}
Positive gaps indicate systematic overshoots requiring additional policy intervention beyond carbon pricing alone.

\subsection{Scenarios}

We implement three NGFS Phase V (November 2024) carbon price trajectories using MESSAGEix-GLOBIOM 2.0 for Other Pacific Asia \citep{NGFS2024}, expressed in US\$2024: \textbf{Net Zero 2050} (\$383/tCO$_2$ in 2030, \$638 in 2050, consistent with 1.5°C); \textbf{Below 2°C} (\$71/\$166, moderate ambition); and \textbf{NDCs} (\$118/\$130, current pledges). Free allocation declines from 8.5 MtCO$_2$ (2025) to 4.2 (2030) and 1.0 (2050), approximating K-ETS reform. Sensitivities test:
\begin{itemize}[leftmargin=*]
  \item \textbf{CCUS availability:} We solve the Net Zero scenario with CCUS deployment disabled to assess reliance on carbon capture for budget compliance.
  \item \textbf{Hydrogen costs:} We test an optimistic case with hydrogen costs 20\% below baseline to examine price thresholds for hydrogen adoption.
  \item \textbf{Scrap constraints:} We vary scrap availability limits to assess circularity pathway contributions.
  \item \textbf{Discount rates:} We test alternative rates (3\% and 7\%) to assess sensitivity to time preference.
\end{itemize}

% ===== 3. Data =====
\section{Data}

Five production routes model POSCO's 37.5 Mt/y capacity: BF-BOF (2.1 tCO$_2$/t, \$1,000/tpy CAPEX, 4 Mt/y modules, 40-year lifetime); BF-BOF+CCUS (0.42 tCO$_2$/t, \$1,400/tpy, 80\% capture, 2027+ availability); Scrap-EAF (0.15 tCO$_2$/t, \$800/tpy, 2 Mt/y, scrap-constrained at 35\% max share); NG-DRI-EAF (0.8 tCO$_2$/t, \$1,800/tpy); H$_2$-DRI-EAF (0.2 tCO$_2$/t, \$2,500/tpy, 45 kg H$_2$/t, 2030+ availability). Demand grows from 37.5 Mt (2025) to 39.2 Mt (2035), declining to 35.8 Mt (2050). Input costs (2024 USD): iron ore \$110/t → \$100/t (2050); coking coal \$180/t → \$160/t; scrap \$420/t → \$480/t; electricity \$75/MWh; hydrogen \$4.50/kg (2030) → \$2.80/kg (2050). All costs discounted at 5\%. Tables \ref{tab:tech-costs} and \ref{tab:tech-intensity} detail parameters.

\begin{table}[ht]
  \centering
  \caption{Technology cost and capacity assumptions}
  \label{tab:tech-costs}
  \begin{threeparttable}
  \begin{tabular}{@{}lccc@{}}
    \toprule
    Route & Unit capacity (Mt/y) & CAPEX (USD/tpy) & Fixed OPEX (USD/tpy) \\
    \midrule
    BF--BOF & 5.0 & 1{,}000 & 100 \\
    BF--BOF+CCUS & 5.0 & 1{,}400 & 150 \\
    FINEX--BOF & 3.0 & 1{,}200 & 120 \\
    Scrap--EAF & 2.0 & 800 & 80 \\
    NG--DRI--EAF & 2.5 & 1{,}800 & 180 \\
    H$_2$--DRI--EAF & 2.0 & 2{,}500 & 200 \\
    HyREX & 1.5 & 3{,}000 & 250 \\
    \bottomrule
  \end{tabular}
  \begin{tablenotes}
    \footnotesize
    \item Notes: Parameters sourced from the engineering data pack (\texttt{data/posco\_parameters\_consolidated.xlsx}; see \citet{MaterialEconomics2019,prammer2021steel,kuramochi2018beyond} for benchmark values). Unit capacities correspond to discrete module sizes in the optimisation. CAPEX/OPEX in real 2024 USD. Scope~1 emission factors and feedstock intensities are reported in Table~\ref{tab:tech-intensity}.
  \end{tablenotes}
  \end{threeparttable}
\end{table}

\begin{table}[ht]
  \centering
  \caption{Scope-1 emission factors and feedstock intensities by technology route}
  \label{tab:tech-intensity}
  \begin{threeparttable}
  \begin{tabular}{@{}lcccccc@{}}
    \toprule
    \multirow{2}{*}{Route} & Scope-1 EF & \multicolumn{5}{c}{Feedstock and energy intensity (per t crude steel)} \\
    \cmidrule(lr){3-7}
     & (tCO$_2$/t) & Iron ore (t) & Scrap (t) & Natural gas (GJ) & Electricity (MWh) & H$_2$ (kg) \\
    \midrule
    BF--BOF & 2.10 & 1.50 & 0.10 & 0.50 & 0.60 & 0 \\
    BF--BOF+CCUS & 2.10\tnote{a} & 1.50 & 0.10 & 0.50 & 0.80 & 0 \\
    FINEX--BOF & 1.95 & 1.40 & 0.15 & 0.40 & 0.65 & 0 \\
    Scrap--EAF & 0.45 & 0.05 & 1.05 & 1.80 & 0.50 & 0 \\
    NG--DRI--EAF & 1.20 & 1.30 & 0.20 & 12.00 & 0.70 & 0 \\
    H$_2$--DRI--EAF & 0.15 & 1.25 & 0.15 & 0.00 & 2.20 & 45 \\
    HyREX & 0.10 & 1.20 & 0.10 & 0.00 & 2.50 & 50 \\
    \bottomrule
  \end{tabular}
  \begin{tablenotes}
    \footnotesize
    \item Notes: Intensities drawn from the \texttt{process\_intensity} and \texttt{ef\_scope1} sheets in \texttt{data/posco\_parameters\_consolidated.xlsx}. Figures reflect steady-state operation with 90\% utilisation.
    \item[a] CCUS routes assume gross emissions of 2.10~tCO$_2$/t with 80\% capture efficiency; the net factor used in the model is 0.42~tCO$_2$/t.
  \end{tablenotes}
  \end{threeparttable}
\end{table}

% ===== 4. Results =====
\section{Results}

Table~\ref{tab:scenario-comparison} summarizes key findings across scenarios. Even aggressive Net Zero pricing (peaking at \$638/tCO$_2$) fails to achieve budget compliance, though it dramatically outperforms moderate trajectories.

\subsection{Emission reductions and budget overshoots}

Net Zero pricing drives emissions from 81 MtCO$_2$ (2025) to 23 MtCO$_2$ (2050), a 71\% decline (Figure~\ref{fig:scope1-by-scenario}). Below 2°C achieves 53 MtCO$_2$ (2050); NDCs plateau at 71 MtCO$_2$. However, cumulative emissions (2025-2050) yield systematic budget overshoots: Net Zero 1,146 MtCO$_2$ (+3.2\% vs 1,110 MtCO$_2$ budget); Below 2°C 1,981 MtCO$_2$ (+78\%); NDCs 1,978 MtCO$_2$ (+78\%). Even the most ambitious scenario narrowly misses compliance; moderate pricing catastrophically fails.

\begin{figure}[!t]
  \centering
  \includegraphics[width=0.8\linewidth]{scope1_by_scenario}
  \caption{Scope~1 emissions by scenario, including Net Zero (No CCUS) sensitivity.}
  \label{fig:scope1-by-scenario}
\end{figure}

Free allocation shields most emissions: only 2.5\% of Net Zero cumulative emissions face ETS liability, translating to an effective price of \$9/tCO$_2$ averaged over 2025-2050. Without CCUS, Net Zero emissions rise to 1,169 MtCO$_2$ (+5.3\% overshoot), with 2050 endpoint at 29 MtCO$_2$ (26\% higher than CCUS-enabled).

\subsection{Technology deployment thresholds}

Technology deployment exhibits sharp thresholds (Figures~\ref{fig:technology-transition}, \ref{fig:production-mix}, \ref{fig:emissions-pathways}). Scrap-EAF remains <5\% until prices exceed \$110/tCO$_2$, then jumps to 23\% as discrete 9 Mt modules are built, capping at 35\% due to scrap availability \citep{POSCO2023SR}. CCUS enters at \$165/tCO$_2$, reaching 51\% of 2050 production under Net Zero. Below 2°C peaks at \$240/tCO$_2$—insufficient for widespread CCUS (64\% remains unabated BF-BOF). NDCs (\$130/tCO$_2$ ceiling) trigger no structural change (>95\% BF-BOF through 2050).

\begin{figure}[!t]
  \centering
  \includegraphics[width=0.8\linewidth]{technology_transition}
  \caption{Technology shares by scenario.}
  \label{fig:technology-transition}
\end{figure}

\begin{figure}[!t]
  \centering
  \includegraphics[width=0.8\linewidth]{production_mix_evolution}
  \caption{Production mix by technology route (Mt/year).}
  \label{fig:production-mix}
\end{figure}

\begin{figure}[!t]
  \centering
  \includegraphics[width=0.85\linewidth]{emissions_pathways}
  \caption{Annual Scope~1 emissions (left) and emissions intensity (right).}
  \label{fig:emissions-pathways}
\end{figure}

Hydrogen DRI remains uneconomic at baseline costs (\$4.5/kg declining to \$2.8/kg), never deployed even at \$638/tCO$_2$ \citep{MaterialEconomics2019,demailly2018european}. Only when CCUS is disabled does H$_2$-DRI capture 41\% (2050), achieving 1,169 MtCO$_2$ cumulative (+5.3\% overshoot) at \$815/t average cost—worse than CCUS-enabled pathways on both emissions and economics. ETS payments quadruple (\$43 billion vs \$9 billion) without CCUS (Figure~\ref{fig:ets-logic}).

\begin{figure}[!t]
  \centering
  \includegraphics[width=0.85\linewidth]{ets_cost_logic}
  \caption{ETS cost mechanics for Net Zero (No CCUS). Top: gross emissions, free allocation, prices; bottom: net liabilities and payments.}
  \label{fig:ets-logic}
\end{figure}

\subsection{Cost competitiveness under high carbon prices}

Ambitious pricing reduces total costs through capital substitution. Net Zero delivers \$791 billion (2025-2050), \$810/t average—lower than Below 2°C (\$891/t) and NDCs (\$861/t). NDCs accumulate \$59.7 billion ETS costs (7.1\% of total) paying recurring fees; Net Zero redirects cash to CCUS/EAF investment, trimming compliance to \$9.3 billion (1.2\%). Marginal abatement cost: Net Zero saves \$64/tCO$_2$ versus NDCs; Below 2°C requires \$108/tCO$_2$ incremental support (Figure~\ref{fig:ets-costs}).

\begin{figure}[!t]
  \centering
  \includegraphics[width=0.8\linewidth]{ets_cost_by_scenario}
  \caption{Annual ETS compliance costs by scenario.}
  \label{fig:ets-costs}
\end{figure}

Table~\ref{tab:scenario-comparison} consolidates findings across scenarios. Even breakthrough hydrogen costs (\$2.5/kg by 2035) yield minimal shifts: cumulative emissions 1,144 MtCO$_2$ (+3.0\% overshoot), nearly identical to baseline. Capital intensity, scrap limits, and pellet feedstock constraints—not H$_2$ molecule prices—bind technology choices \citep{MaterialEconomics2019}.

\section{Discussion}

Our results reveal a critical adequacy gap: even under ambitious NGFS Net Zero pricing ($383/tCO$_2$ by 2030, $638 by 2050), POSCO overshoots its sectoral carbon budget by 5.3\% when forced to rely on hydrogen pathways. Moderate price scenarios catastrophically fail with +78\% budget overshoots. This finding demonstrates that carbon pricing can work, but only under highly ambitious trajectories paired with institutional reforms.

\subsection{Carbon pricing threshold effects}

The optimization identifies clear price thresholds triggering technology deployment: scrap-EAF expansion responds to prices above \$110/tCO$_2$, while hydrogen DRI requires sustained prices exceeding \$350-400/tCO$_2$ to achieve competitiveness. Without CCUS availability, Net Zero pricing drives H$_2$-DRI to 41\% of 2050 output, validating POSCO's strategic bet on hydrogen steelmaking. However, this pathway achieves only 1,169 MtCO$_2$ cumulative emissions—5.3\% above the sectoral budget—despite optimal technology deployment.

This threshold structure creates critical non-linearities. Gradualist price trajectories that rise slowly from low levels risk remaining perpetually below switching thresholds, generating minimal emission reductions despite non-trivial carbon costs. By contrast, credibly high price floors trigger preemptive investment responses: forward-looking firms anticipate future liabilities and commit capital to low-carbon technologies before those high prices fully materialize \citep{fowlie2016carbon}. Our finding that Net Zero pricing reduces average production costs to \$815/t (versus \$861/t under NDCs) demonstrates this dynamic—early aggressive pricing enables capital substitution away from recurring compliance payments.

The adequacy gap persists across three dimensions. First, nominal prices must rise higher in later periods (\$650-700/tCO$_2$ in mid-2040s) to fully activate hydrogen deployment and maximize scrap utilization. Second, free allocation substantially dampens marginal incentives: under Net Zero, only 2.5\% of cumulative emissions trigger actual ETS liability, translating to an effective carbon price of just \$9/tCO$_2$ averaged over the planning horizon. Third, infrastructure readiness proves essential—hydrogen pathways require pellet facilities, pipeline networks, and low-cost renewable electricity that Korea has not yet secured at scale.

\subsection{Institutional barriers}

Korea's free allocation regime remains the binding constraint on emission reductions. Historical practice allocates 95\% of allowances without charge, effectively severing the link between emissions and costs \citep{kim2021kets}. This protection aims to preserve competitiveness and prevent carbon leakage, but creates a self-defeating dynamic: by muting abatement incentives, free allocation delays the technology investments needed for long-term competitiveness under tightening global climate policy \citep{sartor2012benchmark}.

Path dependency compounds these barriers. Korea's integrated steel mills represent tens of billions in sunk capital, with blast furnaces operating 25-40 years. Once installed, marginal production costs remain low relative to alternative technologies, creating powerful lock-in \citep{MaterialEconomics2019}. Concentrated industry structure amplifies this inertia—POSCO accounts for 70\% of domestic production, wielding substantial policy influence that historically secured protection from carbon costs.

Breaking this gridlock requires fundamental reframing: from narrow carbon pricing toward comprehensive industrial strategy addressing infrastructure gaps, coordinating complementary policies, and establishing credible commitment devices.

\subsection{Policy package for hydrogen pathway}

Achieving budget compliance demands integrated reforms across five domains:

\textbf{Binding price floors.} Korea should legislate (not administratively set) a floor price rising from \$80/tCO$_2$ (2025) to \$150 (2030), \$280 (2040), and \$500+ (2045)—exceeding NGFS Net Zero by 20\% in later periods to close the residual budget gap. This provides the legal certainty required for irreversible 25-40 year capital commitments. The UK Carbon Price Floor demonstrates that such mechanisms successfully accelerate technology transitions \citep{Green2021}.

\textbf{Accelerated free allocation phase-out.} Current 2-3\% annual reductions must accelerate to 5-7\% starting immediately, achieving full auctioning by 2035. This ensures firms face full marginal carbon costs before critical blast furnace relining decisions in the early 2030s. Competitiveness concerns should be addressed through carbon border adjustments mirroring the EU's CBAM—not blanket free allocation that undermines abatement incentives.

\textbf{Infrastructure as public good.} Hydrogen pathways require coordinated investment in systems exhibiting public good characteristics: scrap collection networks (targeting 85-90\% recovery versus current 65\%), pellet production facilities for H$_2$-DRI feedstock, and transmission infrastructure delivering low-cost renewable electricity to industrial loads. Government should directly develop these networks following Norway's Northern Lights model for CO$_2$ transport—operating as regulated utilities offering transparent access tariffs \citep{IEA2020steel}.

\textbf{Carbon Contracts for Difference.} CCfDs bridge the mismatch between 25-40 year asset lifetimes and shorter policy credibility horizons. Government guarantees a strike carbon price for qualifying production over 10-15 years; if ETS prices fall short, government compensates the difference; if they exceed, producers refund excess. Germany's Climate Protection Contracts demonstrate viability, with steel sector auctions securing strike prices around €150-180/tCO$_2$ \citep{Neuhoff2019CCfD}. For Korea, we recommend CCfD programs targeting hydrogen DRI demonstration plants (strike prices \$220-280/tCO$_2$) and scrap-EAF expansion (\$80-110/tCO$_2$), with contract volumes aligned to our optimized pathway.

\textbf{Performance standards.} Technology-neutral emission intensity ceilings provide regulatory backstops ensuring progress despite price volatility. We recommend declining thresholds: 2.1 tCO$_2$/t (current) tightening to 1.2 (2035), 0.6 (2042), and 0.3 (2050). Green public procurement for infrastructure projects would create guaranteed demand for low-carbon steel, reducing first-mover market risk.

The fiscal requirements are manageable: ETS auction revenues (\$8-12 billion annually by 2035), CCfD contracts (\$2-3 billion peak), and infrastructure co-investment (\$1-2 billion annually) total \$11-17 billion per year during 2030-2040—equivalent to 0.5-0.8\% of projected GDP. Much of this generates offsetting revenues through ETS payments and infrastructure asset returns.

\subsection{Broader implications}

This research illuminates fundamental limitations in relying on carbon pricing alone for industrial decarbonization. The persistent budget overshoots—occurring even under favorable assumptions and ambitious trajectories—suggest systematic adequacy gaps in capital-intensive sectors with lumpy investments, long asset lifetimes, and limited short-run substitution possibilities.

The steel sector may represent an extreme case, but underlying challenges extend across energy-intensive manufacturing. Cement, chemicals, and aluminum share similar characteristics: capital-intensive long-lived technologies, wholesale infrastructure replacement requirements, and competitive pressures constraining price pass-through. If carbon pricing struggles in Korea's concentrated, well-regulated steel sector—where policy leverage is maximal—prospects for broader industrial decarbonization look questionable.

The policy implication is not to abandon carbon pricing but to embed it within comprehensive strategies addressing its limitations: institutional reforms transmitting prices to investment decisions, infrastructure provision making low-carbon technologies physically feasible, and risk-sharing mechanisms bridging policy-asset lifetime mismatches. This integrated approach resembles successful renewable energy transitions—feed-in tariffs, grid infrastructure investments, and R\&D support—rather than idealized market mechanisms in institutional vacuums \citep{bataille2018role}.

Future research should incorporate uncertainty through stochastic optimization, expand system boundaries to include Scope 3 emissions from hydrogen production, examine firm and regional heterogeneity, integrate with macroeconomic modeling, extend time horizons beyond 2050, and pursue comparative analysis across countries with different industrial structures.

The transition to climate-neutral steel represents a defining industrial challenge. Korea possesses favorable conditions—concentrated structure, strong state capacity, ambitious commitments—yet announced trajectories remain inadequate. Closing this gap requires moving beyond rhetorical commitments toward comprehensive strategies combining credible pricing, accelerated reforms, strategic infrastructure, and targeted risk-sharing.

The adequacy gap reflects not technological or economic constraints but institutional and political barriers. Required technologies exist or are maturing. Economic costs prove manageable when high carbon prices enable capital substitution. What remains is political will to confront incumbent interests, coordinate complex reforms, and maintain commitment through electoral cycles. Whether Korea can summon that will may prove the binding constraint on climate stabilization.

\section{Limitations}\label{sec:limitations}

While this study provides detailed insights into POSCO's optimal decarbonization pathways, several limitations should be acknowledged. First, the model assumes perfect foresight regarding technology costs, carbon prices, and demand trajectories, which may not reflect real-world decision-making under uncertainty. Second, the analysis focuses on a single firm and may not capture broader industry dynamics, including competition effects, supply chain interactions, and technology spillovers across Korean steel producers. Third, the demand pathway is treated as exogenous, whereas carbon pricing and technology transitions could endogenously affect steel consumption patterns through price pass-through and material substitution.

The model also abstracts from several technical and regulatory complexities. Blast furnace relining schedules are approximated rather than explicitly modeled, potentially affecting the precise timing of capacity retirements. Product quality constraints between routes are simplified, and the analysis excludes Scope 3 emissions and lifecycle impacts of hydrogen production. CCUS performance is represented by a deterministic 80\% capture rate with full transport and storage readiness; real-world deployment risk, reservoir availability, and monitoring obligations could materially change the cost and emissions outcomes, so the CCUS-heavy Net Zero portfolio should be interpreted as contingent on aggressive infrastructure delivery. Finally, the study does not model potential complementary policies such as green procurement standards, R\&D subsidies, or international carbon border adjustments, which could significantly alter the investment landscape.

% ===== 7. Conclusion =====
\section{Conclusion}

Can Korea's carbon pricing align POSCO's investment incentives with sectoral carbon budgets? Our mixed-integer optimization across three NGFS Phase V scenarios (November 2024) delivers a nuanced answer: under Net Zero pricing (\$638/tCO$_2$ by 2050), hydrogen pathways achieve 41\% of 2050 output when CCUS is unavailable, reaching 1,169 MtCO$_2$ cumulative emissions—just 5.3\% above the sectoral budget of 1,110 MtCO$_2$. Moderate trajectories catastrophically fail with +78\% overshoots, demonstrating that half-measures prove insufficient.

Three core findings challenge prevailing assumptions. First, ambitious carbon pricing works—Net Zero pricing nearly achieves budget compliance (+5.3\%) while reducing production costs to \$815/t versus \$861/t under NDCs. High credible prices trigger preemptive capital investment in low-carbon technologies, substituting upfront expenditure for recurring compliance payments. This cost competitiveness validates POSCO's hydrogen strategy but requires prices sustained above \$350-400/tCO$_2$.

Second, technology complementarity matters. The No-CCUS sensitivity forcing hydrogen expansion demonstrates that H$_2$-DRI alone cannot deliver both budget compliance and cost-effectiveness simultaneously—emissions overshoot by 5.3\% despite 41\% hydrogen deployment. Korea cannot choose between "green hydrogen" or "CCUS" as singular strategies; both prove necessary alongside aggressive scrap-EAF deployment.

Third, institutional barriers bind more than technology constraints. Free allocation shields 94\% of Net Zero emissions from actual carbon costs, translating to an effective price of just \$9/tCO$_2$ averaged over 2025-2050. Infrastructure gaps—pellet facilities, renewable electricity transmission, scrap collection networks—constrain physical feasibility regardless of price signals.

Policy implications are clear: achieving compliance requires comprehensive industrial strategy, not incremental carbon price adjustments. Five priority reforms: (1) legislated price floors reaching \$500/tCO$_2$ by 2045; (2) accelerated free allocation phase-out achieving full auctioning by 2035; (3) government provision of infrastructure exhibiting public good characteristics; (4) Carbon Contracts for Difference bridging policy-asset lifetime mismatches; (5) technology-neutral performance standards establishing regulatory backstops. Fiscal requirements total \$11-17 billion annually during 2030-2040 (0.5-0.8\% GDP)—manageable given offsetting ETS revenues.

The broader implication: carbon pricing alone proves systematically inadequate for capital-intensive sectors with lumpy investments and long asset lifetimes. If pricing struggles in Korea's concentrated, well-regulated steel sector where policy leverage is maximal, prospects for broader industrial decarbonization through market mechanisms alone look questionable. The policy prescription is not to abandon carbon pricing but to embed it within comprehensive strategies addressing institutional barriers, infrastructure gaps, and investment risk.

The adequacy gap documented here reflects political economy constraints, not technological limits. Required technologies exist or are maturing; economic costs prove manageable. What remains is political will to confront incumbent interests, coordinate complex reforms, and maintain commitment through electoral cycles. Whether Korea can summon that will may prove the binding constraint on climate stabilization—and a test case for whether advanced industrial economies can reconcile deep decarbonization with sustained prosperity.
% ===== Tables =====
\begin{table}[ht]
  \centering
  \caption{Key model assumptions and baseline parameter values (real USD 2024)}
  \label{tab:assumptions}
  \begin{threeparttable}
  \begin{tabular}{@{}llc@{}}
    \toprule
    Category & Parameter & Value/Path \\
    \midrule
    \multirow{3}{*}{Economic} & Discount rate ($\rho$) & 5\% (baseline); 3\% (sensitivity) \\
    & Capacity utilization ($\mu$) & 90\% maximum \\
    & Model horizon & 2025--2050 (26 years) \\
    \midrule
    \multirow{4}{*}{Technology} & BF--BOF unit capacity & 4.0 Mt/y \\
    & EAF unit capacity & 2.0 Mt/y \\
    & CCUS capture efficiency ($\eta^{CCUS}$) & 80\% \\
    & H$_2$-DRI earliest deployment & 2030 \\
    \midrule
    \multirow{3}{*}{Carbon pricing} & Net Zero 2050 (2030/2050) & \$130/\$250 per tCO$_2$ \\
    & Below 2°C (2030/2050) & \$80/\$185 per tCO$_2$ \\
    & NDCs (2030/2050) & \$25/\$75 per tCO$_2$ \\
    \midrule
    \multirow{3}{*}{ETS allocation} & 2025 baseline & 8.5 MtCO$_2$/y \\
    & 2030 (NDC target) & 4.2 MtCO$_2$/y \\
    & 2050 phase-out & 1.0 MtCO$_2$/y \\
    \midrule
    \multirow{3}{*}{Demand} & Initial (2025) & 37.5 Mt/y \\
    & Peak (2035) & 39.2 Mt/y \\
    & Final (2050) & 35.8 Mt/y \\
    \midrule
    \multirow{3}{*}{Emission factors} & BF--BOF & 2.1 tCO$_2$/t steel \\
    & NG-DRI--EAF & 0.8 tCO$_2$/t steel \\
    & H$_2$-DRI--EAF & 0.2 tCO$_2$/t steel \\
    \bottomrule
  \end{tabular}
  \end{threeparttable}
\end{table}
\begin{table}[ht]
  \centering
  \caption{Scenario comparison: Carbon price trajectories, emissions and cost outcomes}
  \label{tab:scenario-comparison}
  \begin{threeparttable}
  \begin{tabular}{@{}lccc@{}}
    \toprule
    Metric & Net Zero 2050 & Below 2$^\circ$C & NDCs \\
    \midrule
    \multicolumn{4}{l}{\textbf{Carbon price path}} \\
    Carbon price 2030 (USD/tCO$_2$) & 150 & 80 & 40 \\
    Carbon price 2050 (USD/tCO$_2$) & 450 & 240 & 100 \\
    \midrule
    \multicolumn{4}{l}{\textbf{Emissions and carbon budget}} \\
    Cumulative Scope~1 emissions (MtCO$_2$) & 1{,}154 & 1{,}365 & 1{,}981 \\
    Budget overshoot (MtCO$_2$) & +44 & +255 & +871 \\
    Budget overshoot (\%) & +4.0 & +23.0 & +78.4 \\
    Budget compliant? & \textbf{No} & \textbf{No} & \textbf{No} \\
    \midrule
    \multicolumn{4}{l}{\textbf{Production mix (2050)}} \\
    Blast furnace share (\%) & 13.3 & 23.5 & 94.9 \\
    Scrap-EAF share (\%) & 86.7 & 76.5 & 5.1 \\
    Hydrogen/CCUS share (\%) & 0.0 & 0.0 & 0.0 \\
    First year scrap $\ge$50\% share & 2033 & 2035 & --- \\
    \midrule
    \multicolumn{4}{l}{\textbf{Economic outcomes}} \\
    Total system cost (2025--2050, billion USD) & 785.8 & 806.7 & 841.7 \\
    Cost vs. NDC (billion USD) & $-$55.9 & $-$35.0 & 0.0 \\
    Average cost per tonne steel (USD/t) & 804 & 825 & 861 \\
    Cost per tonne CO$_2$ avoided (USD/tCO$_2$) & $-$67.7 & $-$56.9 & --- \\
    \midrule
    \multicolumn{4}{l}{\textbf{ETS exposure}} \\
    Net ETS-liable emissions (MtCO$_2$) & 36.8 & 247.5 & 863.0 \\
    Cumulative ETS cost (billion USD) & 3.8 & 24.7 & 59.7 \\
    \bottomrule
  \end{tabular}
  \begin{tablenotes}
    \footnotesize
    \item Notes: Costs expressed in real 2024 USD. Carbon budget equals 1{,}110~MtCO$_2$ (POSCO allocation for 2025--2050). Negative cost values indicate savings relative to the NDC trajectory because higher carbon prices shift expenditure from ETS payments to scrap-based electrification investments. Source: optimisation outputs in \texttt{outputs/analysis}.
  \end{tablenotes}
  \end{threeparttable}
\end{table}


\begin{table}[ht]
  \centering
  \caption{Technology transition milestones in scrap-based electrification}
  \label{tab:technology-thresholds}
  \begin{threeparttable}
  \begin{tabular}{@{}lccc@{}}
    \toprule
    Scenario & Year scrap share $\ge$50\% & Carbon price that year (USD/tCO$_2$) & Scrap share in 2050 (\%) \\
    \midrule
    Net Zero 2050 & 2033 & 195 & 86.7 \\
    Below 2$^\circ$C & 2035 & 120 & 76.5 \\
    NDCs & --- & --- & 5.1 \\
    \bottomrule
  \end{tabular}
  \begin{tablenotes}
    \footnotesize
    \item Notes: Shares refer to POSCO's production mix from optimisation outputs. Scrap share denotes the proportion of steel produced via electric arc furnaces using scrap feedstock.
  \end{tablenotes}
  \end{threeparttable}
\end{table}


\begin{table}[ht]
  \centering
  \caption{Incremental cost of abatement relative to the NDC trajectory}
  \label{tab:cost-abatement}
  \begin{threeparttable}
  \begin{tabular}{@{}lccc@{}}
    \toprule
    Comparison & $\Delta$Cost (billion USD) & $\Delta$Emissions (MtCO$_2$) & Cost per tCO$_2$ (USD) \\
    \midrule
    Net Zero 2050 $-$ NDCs & $-50.4$ & $-790.9$ & $-63.7$ \\
    Below 2$^\circ$C $-$ NDCs & $+29.0$ & $-267.4$ & $+108.4$ \\
    \bottomrule
  \end{tabular}
  \begin{tablenotes}
    \footnotesize
    \item Notes: Negative cost values indicate system-wide savings relative to the NDC scenario. Emission differences computed over 2025--2050 cumulative Scope~1 emissions. All monetary figures in real 2024 USD.
  \end{tablenotes}
  \end{threeparttable}
\end{table}

\begin{table}[ht]
  \centering
  \caption{Policy recommendations to close the carbon pricing adequacy gap}
  \label{tab:policy-matrix}
  \begin{threeparttable}
  \begin{tabular}{@{}p{3.2cm}p{3.2cm}p{2.2cm}p{5.2cm}@{}}
    \toprule
    Policy instrument & Primary target & Timeline & Expected impact \\
    \midrule
    Carbon price floor aligned with NGFS Net Zero & K-ETS allowance auctions & 2026--2035 & Guarantees minimum prices rising to USD\,130/tCO$_2$ by 2030, accelerating low-carbon investments and closing the remaining 80~MtCO$_2$ overshoot. \\
    Accelerated free-allocation phase-out & Emissions-intensive trade-exposed firms & 2026--2035 & Reduces free allocation by 5--7 percentage points per year, exposing firms to full marginal carbon costs before major reinvestment decisions. \\
    Scrap quality and logistics programme & Scrap collectors, steelmakers & 2025--2032 & Expands high-grade scrap supply and processing, sustaining the 35\% electric-arc share achieved in Net Zero and enabling deeper electrification if scrap limits are relaxed. \\
    Industrial power contracts for EAF operators & KEPCO and large consumers & 2025--2030 & Provides predictable low-carbon electricity tariffs, mitigating operating-cost volatility as EAF output rises. \\
    Conditional hydrogen and CCUS support & Emerging technology developers & 2025--2035 & Links public funding to cost-reduction milestones; prevents subsidy lock-in for technologies that remain uncompetitive relative to scrap routes. \\
    International carbon pricing coordination & G20 steel-producing economies & 2026 onward & Aligns carbon price expectations (USD\,150--200/tCO$_2$ by 2035) to manage leakage risks as Korea tightens K-ETS. \\
    Carbon border adjustment engagement & Ministry of Trade, Industry and Energy & 2025--2028 & Coordinates with CBAM jurisdictions to protect exports while rewarding verified low-carbon steel production. \\
    \bottomrule
  \end{tabular}
  \begin{tablenotes}
    \footnotesize
    \item Notes: Recommendations derive from the optimisation results and political economy assessment in the discussion section. Timeline reflects first implementation year through full effect. Expected impacts summarise the mechanism by which each policy supports carbon budget compliance.
  \end{tablenotes}
  \end{threeparttable}
\end{table}


\section*{Data availability}
All input data are derived from publicly available sources cited in the manuscript. Model code and processed scenario outputs (CSV files in \texttt{outputs/analysis/}) will be archived on Zenodo together with this paper and are currently accessible in the project repository (\url{https://github.com/jinsupark/opt\_posco}).

\section*{Funding}
This research received no specific grant from any funding agency, public, commercial, or not-for-profit sectors.

\section*{Conflict of interest}
The author declares no competing financial or personal interests that could have appeared to influence the work reported in this paper.

\section*{Acknowledgements}
The author thanks colleagues at PLANiT Institute for feedback on the optimisation model and policy framing; any remaining errors are the author’s own.


% ===== Bibliography =====
\section*{References}
\begin{thebibliography}{99}

\bibitem[Ahn et al.(2021)]{Ahn2021} Ahn, J., Woo, J., Lee, Y.K. (2021). Optimal transition pathways for Korean steel industry under carbon constraints: A mixed-integer programming approach. \textit{Journal of Cleaner Production}, 308, 127358.

\bibitem[Bataille et al.(2009)]{bataille2018role} Bataille, C., Åhman, M., Neuhoff, K., Nilsson, L.J., Fischedick, M., Lechtenböhmer, S., ... Sartor, O. (2009). The role of sectoral approaches in a future international climate policy framework. \textit{Climate Policy}, 9(4), 406-424.

\bibitem[Calel \& Dechezleprêtre(2016)]{calel2016innovation} Calel, R., Dechezleprêtre, A. (2016). Environmental policy and directed technological change: Evidence from the European carbon market. \textit{Review of Economics and Statistics}, 98(1), 173-191.

\bibitem[Demailly \& Quirion(2018)]{demailly2018european} Demailly, D., Quirion, P. (2008). European Emission Trading Scheme and competitiveness: A case study on the iron and steel industry. \textit{Energy Economics}, 30(4), 2009-2027.

\bibitem[Fowlie et al.(2016)]{fowlie2016carbon} Fowlie, M., Reguant, M., Ryan, S.P. (2016). Market-based emissions regulation and industry dynamics. \textit{Journal of Political Economy}, 124(1), 249-302.

\bibitem[Gasser et al.(2018)]{gasser2018negative} Gasser, T., Guivarch, C., Tachiiri, K., Jones, C.D., Ciais, P. (2018). Negative emissions physically needed to keep global warming below 2°C. \textit{Nature Communications}, 9(1), 1-7.

\bibitem[Government of the Republic of Korea(2020)]{korea2020carbon} Government of the Republic of Korea (2020). \textit{Korean New Deal for a Green Future}. Presidential Committee on Green Growth, Seoul. Available at \url{https://english.moef.go.kr} (accessed September 2024).

\bibitem[Green(2021)]{Green2021} Green, F. (2021). Does carbon pricing reduce emissions? A review of ex-post analyses. \textit{Environmental Research Letters}, 16(4), 043004.

\bibitem[Griffin et al.(2020)]{Griffin2020} Griffin, P.W., Hammond, G.P., Norman, J.B. (2020). Industrial decarbonisation of the pulp \& paper sector: A UK perspective. \textit{Applied Thermal Engineering}, 134, 152-162.

\bibitem[ICAP(2024)]{ICAP2024} International Carbon Action Partnership (2024). \textit{Korea Emissions Trading System}. ICAP ETS Detailed Information. Seoul. Available at \url{https://icapcarbonaction.com} (accessed September 2024).

\bibitem[IEA(2020)]{IEA2020steel} International Energy Agency (2020). \textit{Iron and Steel Technology Roadmap: Towards More Sustainable Steelmaking}. OECD/IEA, Paris.

\bibitem[NGFS(2024)]{NGFS2024} Network for Greening the Financial System (2024). \textit{NGFS Climate Scenarios for Central Banks and Supervisors -- Phase V}. November 2024. NGFS Secretariat, Paris. Available at \url{https://www.ngfs.net/ngfs-scenarios-portal/} and \url{https://data.ene.iiasa.ac.at/ngfs/} (accessed January 2025).

\bibitem[Jarke \& Perino(2017)]{jarke2017carbon} Jarke, J., Perino, G. (2017). Do renewable energy policies reduce carbon emissions? On caps and inter-industry leakage. \textit{Journal of Environmental Economics and Management}, 84, 102-124.

\bibitem[Kim \& Lee(2021)]{kim2021kets} Kim, S., Lee, K. (2021). The evolution and impact of the Korean Emissions Trading System: 2015-2020 assessment. \textit{Environmental Science \& Policy}, 125, 1-12.

\bibitem[KOSIS(2023)]{KOSIS2023} Korean Statistical Information Service (2023). \textit{Greenhouse Gas Emissions by Economic Activity}. Statistics Korea, Daejeon.

\bibitem[Kuramochi et al.(2018)]{kuramochi2018beyond} Kuramochi, T., H{\"o}hne, N., Schaeffer, M., Cantzler, J., Hare, W., Deng, Y., ... Blok, K. (2018). Ten key short-term sectoral benchmarks to limit warming to 1.5°C. \textit{Climate Policy}, 18(3), 287-305.

\bibitem[Martin et al.(2016)]{martin2016industry} Martin, R., Muûls, M., Wagner, U.J. (2016). The impact of the European Union Emissions Trading System on regulated firms: What is the evidence after ten years? \textit{Review of Environmental Economics and Policy}, 10(1), 129-148.

\bibitem[Material Economics(2019)]{MaterialEconomics2019} Material Economics (2019). \textit{Industrial Transformation 2050: Pathways to Net-Zero Emissions from EU Heavy Industry}. Material Economics, Stockholm.

\bibitem[Matthews et al.(2009)]{matthews2009proportionality} Matthews, H.D., Gillett, N.P., Stott, P.A., Zickfeld, K. (2009). The proportionality of global warming to cumulative carbon emissions. \textit{Nature}, 459(7248), 829-832.

\bibitem[Millar et al.(2017)]{millar2017emission} Millar, R.J., Fuglestvedt, J.S., Friedlingstein, P., Rogelj, J., Grubb, M.J., Matthews, H.D., ... Allen, M.R. (2017). Emission budgets and pathways consistent with limiting warming to 1.5°C. \textit{Nature Geoscience}, 10(10), 741-747.

\bibitem[Otto et al.(2017)]{Otto2017} Otto, A., Robinius, M., Grube, T., Schiebahn, S., Praktiknjo, A., Stolten, D. (2017). Power-to-steel: Reducing CO$_2$ through the integration of renewable energy and hydrogen into the German steel industry. \textit{Energies}, 10(4), 451.

\bibitem[Pauliuk et al.(2013)]{pauliuk2013global} Pauliuk, S., Wang, T., Müller, D.B. (2013). Steel all over the world: Estimating in-use stocks of iron for 200 countries. \textit{Resources, Conservation and Recycling}, 71, 22-30.

\bibitem[Prammer et al.(2021)]{prammer2021steel} Prammer, H., Rust, C., Steinlechner, S. (2021). Steel production with hydrogen: A technical and economic analysis. \textit{International Journal of Hydrogen Energy}, 46(70), 34507-34516.

\bibitem[POSCO Holdings(2023)]{POSCO2023SR} POSCO Holdings (2023). \textit{Sustainability Report 2023}. POSCO Holdings, Pohang. Available at \url{https://www.posco-inc.com}.

\bibitem[Korea Iron \& Steel Association(2024)]{KOSA2024Yearbook} Korea Iron \& Steel Association (2024). \textit{Korean Steel Industry Yearbook 2024}. KOSA, Seoul. Available at \url{https://www.kosa.or.kr} (accessed September 2024).

\bibitem[Raupach et al.(2014)]{raupach2014sharing} Raupach, M.R., Davis, S.J., Peters, G.P., Andrew, R.M., Canadell, J.G., Ciais, P., ... Le Quéré, C. (2014). Sharing a quota on cumulative carbon emissions. \textit{Nature Climate Change}, 4(10), 873-879.

\bibitem[Robiou du Pont et al.(2017)]{robiou2019national} Robiou du Pont, Y., Jeffery, M.L., Gütschow, J., Rogelj, J., Christoff, P., Meinshausen, M. (2017). Equitable mitigation to achieve the Paris Agreement goals. \textit{Nature Climate Change}, 7(1), 38-43.

\bibitem[Rogelj et al.(2019)]{rogelj2019new} Rogelj, J., Forster, P.M., Kriegler, E., Smith, C.J., Séférian, R. (2019). Estimating and tracking the remaining carbon budget for stringent climate targets. \textit{Nature}, 571(7765), 335-342.

\bibitem[Sartor(2012)]{sartor2012benchmark} Sartor, O. (2012). How to interpret the EU ETS benchmark curves? \textit{Climate Policy}, 12(6), 745-765.

\bibitem[Neuhoff et al.(2019)]{Neuhoff2019CCfD} Neuhoff, K., Chiappinelli, O., Gerres, T., Haussner, M., Ismer, R., May, N., Richstein, J.C., Schütze, F. (2019). \textit{Climate Policy for Industrial Innovation: Carbon Contracts for Difference}. Climate Friendly Materials Platform, Berlin.

\bibitem[Richstein(2017)]{Richstein2017CCfD} Richstein, J.C. (2017). \textit{Carbon Contracts for Difference: An Economic Assessment of Policy Instruments for Industrial Decarbonisation}. DIW Berlin Discussion Paper 1623, Berlin.

\bibitem[Ueckerdt et al.(2021)]{ueckerdt2021potential} Ueckerdt, F., Bauer, C., Dirnaichner, A., Everall, J., Sacchi, R., Luderer, G. (2021). Potential and risks of hydrogen-based e-fuels in climate change mitigation. \textit{Nature Climate Change}, 11(5), 384-393.

\bibitem[Vogl et al.(2018)]{Vogl2018} Vogl, V., Åhman, M., Nilsson, L.J. (2018). Assessment of hydrogen direct reduction for fossil-free steelmaking. \textit{Journal of Cleaner Production}, 203, 736-745.

\bibitem[Wang et al.(2021b)]{wang2021carbon} Wang, X., Cai, Y., Xie, Y., Zhao, T. (2021). Carbon pricing in China's national ETS: Insights from municipal carbon pricing pilots. \textit{Energy Policy}, 154, 112288.

\bibitem[Wang et al.(2021a)]{wang2021hydrogen} Wang, P., Ryberg, D.S., Yang, Y., Grube, T., Robinius, M., Stolten, D. (2021). Spatial and temporal assessment of global hydrogen production from renewable resources. \textit{Energy}, 238, 121532.

\bibitem[World Steel(2022)]{worldsteel2022} World Steel Association (2022). \textit{World Steel in Figures 2022}. World Steel Association, Brussels.

\bibitem[Zhang et al.(2022)]{zhang2022steel} Zhang, X., Wang, Y., Li, F., Zhang, Y., Huang, L., Liu, Y. (2022). An integrated assessment of China's steel sector transition: Jointly considering economic growth and carbon mitigation. \textit{Journal of Cleaner Production}, 342, 130909.
\end{thebibliography}
\end{document}
