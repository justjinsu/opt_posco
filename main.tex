\documentclass[preprint,1p,authoryear]{elsarticle}

% ===== Packages =====
\usepackage[T1]{fontenc}
\usepackage[utf8]{inputenc}
\usepackage{lmodern}
\usepackage{amsmath,amssymb}
\usepackage{booktabs,threeparttable,siunitx}
\usepackage{graphicx}
\usepackage{subcaption}
\usepackage{longtable}
\usepackage{multirow}
\usepackage{hyperref}
\usepackage{xcolor}
\usepackage{enumitem}
\usepackage{geometry}
\geometry{margin=1in}
\usepackage{csvsimple}
\usepackage{pgfplotstable}

% Graphics path
\graphicspath{{figures/}}
\journal{Energy Policy}
\biboptions{authoryear}


% ===== Macros =====
\newcommand{\R}{\mathbb{R}}
\newcommand{\N}{\mathbb{N}}
\newcommand{\E}{\mathrm{E}}
\newcommand{\COtwo}{CO$_2$}
\newcommand{\todo}[1]{\textcolor{red}{[TODO: #1]}}
\sisetup{detect-weight=true, detect-family=true, group-separator=\,}

% ===== Title & Authors =====
\begin{document}
\begin{frontmatter}
\title{Testing the limits of carbon pricing: Can Korea's emissions trading system align steel industry investments with national climate targets?}

\author[planit]{Jinsu Park}
\address[planit]{PLANiT Institute, Seoul, Republic of Korea}

% ===== Highlights (Energy Policy requires 3--5 bullet highlights) 

\section*{Highlights}
\begin{itemize}[leftmargin=*]
  \item \textbf{First empirical test} of carbon pricing adequacy: Only NGFS Net Zero 2050 scenario keeps POSCO within Korea's sectoral carbon budget (1,110 MtCO$_2$, 2025-2050).
  \item \textbf{Policy-performance gap revealed}: NDC-consistent carbon prices overshoot carbon budget by 38\%, exposing systematic policy failure in energy-intensive industries.
  \item \textbf{Threshold effects identified}: Carbon prices must reach \$130/tCO$_2$ by 2030 to trigger hydrogen-DRI adoption before 2035, enabling budget compliance.
  \item \textbf{Mixed-integer optimization} of technology transitions across five routes (BF-BOF, CCUS, EAF, NG-DRI, H$_2$-DRI) with realistic investment constraints and lumpy capacity decisions.
  \item \textbf{Clear policy prescription}: Korea must double current K-ETS price trajectory and accelerate free allocation phase-out to align industrial investments with climate targets.
\end{itemize}

% ===== Abstract =====
\begin{abstract}
Can carbon pricing alone drive the massive industrial transformation required for climate neutrality? We test this fundamental policy question using Korea's steel sector—responsible for 10\% of national emissions—as a critical case study. Through mixed-integer optimization of POSCO's technology portfolio under three NGFS carbon price scenarios, we reveal a stark policy-performance gap: only aggressive carbon pricing reaching \$250/tCO$_2$ by 2050 can align profit-maximizing corporate decisions with Korea's sectoral carbon budget of 1,110 MtCO$_2$ (2025-2050). Current policy trajectories systematically fail this test. The NDC-consistent scenario overshoots the carbon budget by 38\% (425 MtCO$_2$), while even the Below 2°C pathway exceeds limits by 16\% (180 MtCO$_2$). Only the Net Zero 2050 scenario—requiring carbon prices of \$130/tCO$_2$ by 2030—delivers budget-compliant emissions of 1,045 MtCO$_2$ through early hydrogen-DRI adoption before 2035. These findings expose the inadequacy of current Korean ETS price signals and demonstrate that incremental carbon price increases will systematically underdeliver on climate commitments. We conclude that Korea must double its carbon price trajectory by 2030 and accelerate free allocation phase-out to ensure policy-target alignment in energy-intensive industries.
\end{abstract}

\begin{keyword}
Steel decarbonization \sep carbon pricing \sep mixed-integer optimization \sep hydrogen DRI \sep CCUS \sep Korea ETS \sep CBAM
\JEL{Q41, Q54, L61}
\end{keyword}
\end{frontmatter}

% ===== 1. Introduction =====
\section{Introduction}

Can market-based climate policies deliver the massive industrial transformation required for net-zero emissions? This fundamental question lies at the heart of contemporary climate policy design, where carbon pricing has emerged as the preferred instrument for decarbonizing energy-intensive industries. Yet despite widespread adoption of emissions trading systems globally, empirical evidence on their adequacy for achieving climate targets remains surprisingly limited \citep{Green2021}. This paper provides the first rigorous test of carbon pricing effectiveness in driving industrial decarbonization consistent with sectoral carbon budgets.

We focus on Korea's steel sector—a critical case study that illuminates broader challenges facing carbon pricing in heavy industry. Steel production is responsible for approximately 7\% of global CO$_2$ emissions \citep{worldsteel2022}, with Korea ranking as the world's sixth-largest producer. The sector's concentration amplifies policy leverage: POSCO, the dominant domestic producer, alone accounts for 10\% of Korea's national greenhouse gas emissions \citep{KOSIS2023}—equivalent to the entire emissions of countries like Belgium or Chile. If carbon pricing cannot drive decarbonization in such concentrated, high-emission industries, its prospects for economy-wide transformation appear dim.. 

The steel sector presents a particularly demanding test case for carbon pricing effectiveness. Deep decarbonization requires wholesale technology transformation from carbon-intensive blast furnace-basic oxygen furnace (BF-BOF) routes to emerging low-carbon alternatives: hydrogen-based direct reduced iron (H$_2$-DRI), carbon capture and storage (CCUS), and electric arc furnaces (EAF) fed by low-carbon electricity \citep{IEA2020steel}. These transitions involve massive capital commitments—often exceeding \$2 billion per plant—with technology lifespans of 25-40 years \citep{MaterialEconomics2019}. The lumpy, irreversible nature of these investments creates powerful path dependencies that carbon pricing must overcome.

Korea's institutional context provides an ideal natural experiment for testing carbon pricing adequacy. The Korean Emissions Trading System (K-ETS), launched in 2015, covers 70\% of national emissions including the steel sector \citep{kim2021kets}. However, generous free allocation has historically shielded steel producers from carbon costs, with POSCO receiving allowances covering 95\% of its emissions \citep{ICAP2024}. This allocation is scheduled to decline in line with Korea's 2030 nationally determined contribution (40\% reduction from 2018 levels) and 2050 carbon neutrality target, gradually exposing steel producers to carbon price signals.

The critical policy question is whether planned carbon price trajectories—reflecting Korea's international climate commitments—can drive the technology transitions required for sectoral carbon budget compliance. Korea's carbon neutrality framework implicitly allocates a finite carbon budget to each economic sector \citep{korea2020carbon}. For steel production, we derive a sectoral budget of approximately 1,850 MtCO$_2$ over 2025-2050, with POSCO's proportional allocation of 1,110 MtCO$_2$. The fundamental test of carbon pricing adequacy is whether profit-maximizing corporate investment decisions remain within this budget constraint.

To examine this question empirically, we develop a mixed-integer linear programming model that optimizes POSCO's technology portfolio under three carbon price scenarios aligned with Network for Greening the Financial System (NGFS) pathways \citep{NGFS2024}. The model minimizes the net present value of total system costs—capital expenditure, operating costs, and carbon costs—subject to technology constraints, feedstock availability, and product quality requirements. By comparing optimal emission pathways against derived carbon budgets, we provide the first rigorous assessment of whether current carbon pricing trajectories can align industrial investment incentives with climate targets.

\textbf{Central Hypothesis:} We hypothesize that carbon pricing aligned with ambitious climate scenarios (NGFS Net Zero 2050, reaching \$250/tCO$_2$ by 2050) represents a necessary but potentially insufficient condition for sectoral carbon budget compliance, while carbon pricing consistent with current policy trajectories (NGFS NDCs, reaching \$75/tCO$_2$ by 2050) will systematically overshoot budget allocations, creating a dangerous policy-performance gap that threatens national climate commitments.

\textbf{Preview of Findings:} Our results provide sobering evidence of carbon pricing limitations under current policy settings. Only the most aggressive carbon price scenario—Net Zero 2050—delivers emissions consistent with sectoral carbon budget constraints. Even the Below 2°C pathway overshoots budget allocations by 16\%, while NDC-consistent carbon pricing leads to a 38\% budget overshoot equivalent to 425 MtCO$_2$ of excess emissions. These findings suggest that current carbon pricing trajectories are fundamentally inadequate for achieving stated climate targets in energy-intensive industries, requiring urgent policy recalibration to close the emerging policy-performance gap.

\section{Literature Review}

This study contributes to three distinct literatures: steel sector decarbonization pathways, carbon pricing effectiveness in heavy industry, and sectoral carbon budget allocation. We identify critical gaps in each stream and position our contribution at their intersection.

\subsection{Steel sector decarbonization: Technology pathways and constraints}

The steel decarbonization literature has evolved from early techno-economic assessments of individual technologies to comprehensive system-level analyses incorporating feedstock constraints, infrastructure requirements, and policy interactions. \citet{IEA2020steel} established the canonical taxonomy of decarbonization routes: (i) carbon management through CCUS retrofits to existing integrated steelworks; (ii) circularity and electrification via scrap-based electric arc furnaces; and (iii) hydrogen-based reduction replacing coking coal with H$_2$ as the primary reductant.

Early studies emphasized technical feasibility and cost comparisons under static assumptions \citep{Vogl2018, Otto2017}. More recent work incorporates dynamic considerations including technology learning \citep{prammer2021steel}, infrastructure co-evolution \citep{ueckerdt2021potential}, and system-level constraints on scrap availability \citep{pauliuk2013global} and renewable hydrogen production \citep{wang2021hydrogen}. \citet{MaterialEconomics2019} provided the first comprehensive roadmap for European steel decarbonization, highlighting the critical role of policy timing in determining optimal technology sequences.

However, existing studies exhibit three key limitations. \textbf{First}, most analyses treat technology adoption as continuous variables, overlooking the lumpy, irreversible nature of blast furnace campaigns and capital stock replacement cycles \citep{Griffin2020}. \textbf{Second}, few studies integrate region-specific policy frameworks, feedstock availability, and infrastructure constraints that critically shape decarbonization pathways \citep{zhang2022steel}. \textbf{Third}, no existing work evaluates decarbonization scenarios against carbon budget constraints derived from national climate commitments, despite this being the ultimate test of policy adequacy.

\subsection{Carbon pricing effectiveness in energy-intensive industries}

The carbon pricing literature has extensively documented price responsiveness in power generation \citep{jarke2021carbon} and oil refining \citep{fowlie2016carbon}, but evidence for energy-intensive manufacturing remains limited and contradictory. \citet{calel2016innovation} found that EU ETS coverage accelerated clean innovation in participating firms, while \citet{martin2016industry} documented substantial emissions leakage through production shifting to unregulated regions.

The steel sector presents unique challenges for carbon pricing effectiveness. \citet{sartor2017benchmark} highlighted how free allocation in EU ETS Phase III eliminated carbon cost exposure for most steel producers, while \citet{neuhoff2012inclusion} showed that even modest carbon prices could trigger investment in energy efficiency. \citet{demailly2018european} found that carbon prices above €50/tCO$_2$ would be required to make hydrogen-based DRI competitive against conventional blast furnaces.

Recent studies have begun examining carbon pricing in emerging market contexts. \citet{kim2021kets} documented the evolution of K-ETS from 2015-2020, showing limited price discovery and minimal industrial restructuring. \citet{wang2021carbon} found similar patterns in China's national ETS pilot programs. However, \textbf{no existing study rigorously tests whether planned carbon price trajectories can drive the scale and timing of technology transitions required for sectoral carbon budget compliance}.

\subsection{Sectoral carbon budgets and allocation mechanisms}

The carbon budget literature originated in climate science, establishing relationships between cumulative emissions and temperature outcomes \citep{matthews2009proportionality}. Recent work has extended these concepts to national and sectoral levels, providing frameworks for allocating global carbon budgets across countries \citep{raupach2014sharing} and economic sectors \citep{gasser2018negative}.

\citet{rogelj2019new} refined global carbon budget estimates consistent with 1.5°C and 2°C temperature targets, while \citet{millar2017emission} highlighted the critical importance of near-term emission trajectories. However, translating global budgets to operational policy frameworks remains challenging. \citet{robiou2019national} proposed methodologies for national carbon budget allocation, while \citet{kuramochi2018beyond} examined sectoral decomposition approaches.

The steel sector's high emissions intensity and limited abatement options create particular challenges for carbon budget allocation. \citet{bataille2018role} estimated that steel production could consume 15-20\% of remaining global carbon budgets under business-as-usual trajectories. \citet{Griffin2020} emphasized the importance of sectoral carbon budgets for guiding industrial policy, but \textbf{no existing work has empirically tested whether current policy instruments can deliver budget-compliant outcomes in energy-intensive sectors}.

\subsection{Research contribution and positioning}

This study addresses critical gaps across all three literature streams by providing the first empirical test of carbon pricing adequacy for sectoral carbon budget compliance in heavy industry. Our contributions include:

\textbf{Methodological innovation:} We develop the first integrated framework combining bottom-up technology optimization with top-down carbon budget constraints, enabling rigorous assessment of policy-performance gaps.

\textbf{Empirical evidence:} We provide the first quantitative assessment of carbon pricing effectiveness in driving technology transitions consistent with climate targets, using Korea's steel sector as a critical case study.

\textbf{Policy insights:} We identify specific carbon price thresholds required for budget compliance and quantify the magnitude of policy reforms needed to close emerging policy-performance gaps.

Our approach bridges engineering-economic optimization with climate policy evaluation, establishing a new framework for assessing industrial decarbonization policies that can be extended to other sectors and jurisdictions.

% ===== 2. Methods: Optimization model =====
\section{Methodology}

\subsection{Conceptual framework and model overview}

We develop an integrated assessment framework combining bottom-up technology optimization with top-down carbon budget evaluation to test carbon pricing adequacy in driving industrial transformation. The framework operates across three analytical levels:

\textbf{Level 1: Technology Portfolio Optimization.} A mixed-integer linear programming (MILP) model determines least-cost technology adoption pathways for POSCO's steel production system (2025-2050) under alternative carbon price scenarios. The model captures lumpy investment decisions, technology lifetimes, and operational constraints.

\textbf{Level 2: Carbon Budget Derivation.} We derive Korea's steel sector carbon budget from national climate commitments using proportional allocation based on current sectoral emissions shares and NDC reduction trajectories.

\textbf{Level 3: Policy-Performance Gap Assessment.} We compare optimal emission pathways from Level 1 against carbon budget constraints from Level 2, quantifying the magnitude of policy-performance gaps across carbon price scenarios.

This multi-level approach enables rigorous testing of our central hypothesis: that current carbon pricing trajectories are inadequate for sectoral carbon budget compliance, requiring policy recalibration to align industrial investment incentives with climate targets.

\subsection{Optimization model formulation}

The core optimization model determines POSCO's optimal technology portfolio by minimizing the net present value of total system costs subject to physical, technological, and policy constraints. The model represents five distinct production routes with realistic investment discreteness and operational constraints.

\subsection{Sets and indices}
\begin{itemize}[leftmargin=*]
    \item $t \in \mathcal{T}$: years, $t = 2025, \dots, 2050$.
    \item $r \in \mathcal{R}$: production routes, including BF--BOF, BF--BOF+CCUS, Scrap-EAF, NG-DRI--EAF, H$_2$-DRI--EAF.
    \item $\mathcal{R}^{CCUS} \subseteq \mathcal{R}$: subset of routes equipped with CCUS technology.
    \item $\mathcal{R}^{H_2} \subseteq \mathcal{R}$: subset of hydrogen-based routes (H$_2$-DRI, HyREX).
\end{itemize}

\subsection{Decision variables}
\begin{itemize}[leftmargin=*]
    \item $B_{r,t} \in \mathbb{Z}_{\ge 0}$: number of units of route $r$ built in year $t$.
    \item $K_{r,t} \in \mathbb{R}_{\ge 0}$: available capacity of route $r$ in year $t$ (Mt/y).
    \item $Q_{r,t} \in \mathbb{R}_{\ge 0}$: annual production from route $r$ in year $t$ (Mt).
    \item $ETS_{t}^{+} \in \mathbb{R}_{\ge 0}$: positive part of net ETS liability in year $t$ (Mt CO$_2$).
\end{itemize}

\subsection{Objective function}

The optimization problem minimizes the net present value of total system costs, incorporating capital expenditure, operating costs, and carbon pricing under Korea's emissions trading system:

\begin{align}
\min_{B,K,Q,ETS^+} \; & \sum_{t \in \mathcal{T}} \delta_t \left[ C^{CAPEX}_t + C^{FixedOM}_t + C^{VarOPEX}_t + C^{ETS}_t \right],
\end{align}

where $\delta_t = (1+\rho)^{-(t-t_0)}$ is the discount factor with real discount rate $\rho$ and base year $t_0 = 2025$. The cost components are:

\begin{align}
C^{CAPEX}_t &= \sum_{r \in \mathcal{R}} B_{r,t} \cdot \kappa_r \cdot c^{capex}_r \cdot 10^6, \label{eq:capex}\\
C^{FixedOM}_t &= \sum_{r \in \mathcal{R}} K_{r,t} \cdot c^{fixom}_r \cdot 10^6, \label{eq:fixom}\\
C^{VarOPEX}_t &= \sum_{r \in \mathcal{R}} Q_{r,t} \cdot \left( \sum_{i \in \mathcal{I}} \alpha_{r,i} \cdot p_{i,t} \right) \cdot 10^6, \label{eq:varopex}\\
C^{ETS}_t &= P^{CO_2}_t \cdot ETS_t^+ \cdot 10^6, \label{eq:ets}
\end{align}

where $\kappa_r$ denotes unit capacity (Mt/y), $c^{capex}_r$ and $c^{fixom}_r$ are technology-specific cost parameters (USD/tpy and USD/tpy/y respectively), $\alpha_{r,i}$ represents input intensity of commodity $i$ for route $r$ (physical units per ton steel), $p_{i,t}$ denotes commodity prices (USD per physical unit), $P^{CO_2}_t$ is the carbon price (USD/tCO$_2$), and $ETS_t^+$ captures net ETS liability (Mt CO$_2$/y).

The variable operating cost formulation in Eq.~\eqref{eq:varopex} explicitly represents major input commodities $\mathcal{I} = \{$iron ore, coking coal, scrap steel, natural gas, electricity, hydrogen, fluxes, alloys$\}$, enabling detailed representation of feedstock substitution across technology routes.

\subsection{Operational and technological constraints}

The optimization is subject to operational, technological, and policy constraints that ensure realistic representation of steel production dynamics:

\textbf{Material balance and production constraints:}
\begin{align}
\sum_{r \in \mathcal{R}} Q_{r,t} &= D_t, \quad \forall t \in \mathcal{T}, \label{eq:demand}\\
Q_{r,t} &\le \mu \cdot K_{r,t}, \quad \forall r \in \mathcal{R}, t \in \mathcal{T}, \label{eq:utilization}\\
K_{r,t} &= K_{r,t-1} + \kappa_r \cdot B_{r,t}, \quad \forall r \in \mathcal{R}, t \in \mathcal{T} \setminus \{t_0\}, \label{eq:capacity}\\
K_{r,t_0} &= K_r^{initial}, \quad \forall r \in \mathcal{R}, \label{eq:initial}
\end{align}

where $D_t$ denotes exogenous steel demand (Mt/y), $\mu$ represents maximum capacity utilization (90\%), and $K_r^{initial}$ specifies initial capacity endowments.

\textbf{Emissions and carbon pricing constraints:}
\begin{align}
ETS_t^+ &\ge \sum_{r \in \mathcal{R}} ef_r^{net} \cdot Q_{r,t} - A_t^{free}, \quad \forall t \in \mathcal{T}, \label{eq:ets_balance}\\
ef_r^{net} &= ef_r^{gross} \cdot (1 - \eta^{CCUS}), \quad \forall r \in \mathcal{R}^{CCUS}, \label{eq:ccus_factor}\\
ef_r^{net} &= ef_r^{gross}, \quad \forall r \in \mathcal{R} \setminus \mathcal{R}^{CCUS}, \label{eq:no_ccus}
\end{align}

where $ef_r^{gross}$ and $ef_r^{net}$ denote gross and net emission factors (tCO$_2$/t steel), $\eta^{CCUS}$ represents CCUS capture efficiency (80\%), and $A_t^{free}$ specifies free allocation under K-ETS declining schedules.

\textbf{Technology deployment and timing constraints:}
\begin{align}
B_{r,t} &= 0, \quad \forall r \in \mathcal{R}^{H_2}, t < 2030, \label{eq:h2_timing}\\
\sum_{r \in \mathcal{R}^{CCUS}} B_{r,t} &= 0, \quad \forall t < 2027, \label{eq:ccus_timing}\\
B_{r,t} &\in \mathbb{Z}_{\ge 0}, \quad \forall r \in \mathcal{R}, t \in \mathcal{T}, \label{eq:integer}
\end{align}

reflecting realistic technology readiness timelines and investment lumpiness.

\subsection{Carbon budget framework and policy-performance gap assessment}

To evaluate carbon pricing adequacy, we derive Korea's steel sector carbon budget from national climate commitments and compare against optimized emission pathways.

\textbf{Sectoral carbon budget derivation:}
Korea's steel sector carbon budget $CB^{steel}$ is derived using proportional allocation based on current sectoral emissions share $\phi^{steel}$ and national emission trajectories consistent with NDC commitments:

\begin{align}
CB^{steel} &= \sum_{t=2025}^{2050} E^{national}_t \cdot \phi^{steel}, \label{eq:budget_total}\\
E^{national}_t &= E^{national}_{2018} \cdot (1 - \beta \cdot \min(1, (t-2018)/12)) \cdot (1 - \gamma \cdot \max(0, (t-2030)/20)), \label{eq:national_trajectory}
\end{align}

where $\beta = 0.4$ represents the 2030 NDC reduction target, $\gamma = 0.6$ captures additional reductions for 2050 carbon neutrality, and $\phi^{steel} = 0.12$ reflects steel's current share of national emissions.

POSCO's carbon budget allocation is:
\begin{align}
CB^{POSCO} = CB^{steel} \cdot \phi^{POSCO} = CB^{steel} \cdot 0.6, \label{eq:posco_budget}
\end{align}

yielding $CB^{POSCO} \approx 1{,}110$ MtCO$_2$ over 2025-2050.

\textbf{Policy-performance gap quantification:}
For each carbon price scenario $s$, we define the policy-performance gap as:

\begin{align}
\text{Gap}_s &= \frac{\sum_{t \in \mathcal{T}} E_{s,t}^{optimal} - CB^{POSCO}}{CB^{POSCO}} \times 100\%, \label{eq:gap}\\
E_{s,t}^{optimal} &= \sum_{r \in \mathcal{R}} ef_r^{net} \cdot Q_{r,t}^*(s), \label{eq:optimal_emissions}
\end{align}

where $Q_{r,t}^*(s)$ represents optimal production levels under scenario $s$. Positive gaps indicate carbon budget overshooting, while negative gaps suggest budget compliance with emission reductions exceeding allocated limits.

\subsection{Carbon price scenarios and free allocation}
We implement three NGFS-based carbon price trajectories—Net Zero 2050, Below 2°C, and NDCs—expressed in USD/tCO$_2$ and converted to KRW for OPEX calculation. Free allocation $A^{free}_t$ is assumed to decline linearly in proportion to Korea's industrial-sector GHG reduction targets for 2030 and 2050.

\subsection{CCUS emission factor calculation}
For routes equipped with CCUS (denoted by $\mathcal{R}^{CCUS}$), the net emission factor is calculated as:
\begin{align}
ef_r^{net} = ef_r^{gross} \cdot (1 - \eta^{CCUS}), \quad \forall r \in \mathcal{R}^{CCUS},
\end{align}
where $ef_r^{gross}$ is the gross emission factor and $\eta^{CCUS}$ is the CCUS capture efficiency (baseline: 80\%).

\subsection{Solution method}
The MILP is implemented in \texttt{Pyomo} and solved using the \texttt{HiGHS} or \texttt{GLPK} solver. The model is run for each carbon price scenario with baseline and optimistic hydrogen cost assumptions. Key outputs include:
\begin{itemize}[leftmargin=*]
  \item Annual production mix by technology route ($Q_{r,t}$)
  \item Capacity building decisions ($B_{r,t}$) and evolution ($K_{r,t}$)
  \item Annual Scope~1 emissions and ETS expenditures
  \item NPV cost decomposition (CAPEX, OPEX, ETS costs)
  \item Hydrogen and electricity demand projections
\end{itemize}

% ===== 3. Data and scenarios =====
\section{Data and scenario design}

This section details the data sources, scenario construction, and parameter calibration underlying our analysis. All monetary values are expressed in real 2024 USD unless otherwise specified.

\subsection{System boundary and company scope}

Our analysis focuses on POSCO's Scope 1 emissions from integrated steel production, covering approximately 37.5 Mt/y of crude steel capacity across major facilities including Pohang and Gwangyang works. The system boundary encompasses iron ore reduction, steelmaking, and energy-related combustion but excludes upstream supply chain emissions (Scope 3) and downstream processing activities.

This boundary choice reflects K-ETS coverage, which applies carbon pricing only to direct emissions from steel production. However, we conduct sensitivity analysis incorporating Scope 2 emissions using Korea's grid decarbonization trajectory to assess broader climate impacts of electricity-intensive routes.

POSCO represents an ideal case study for several reasons: (i) market dominance (60\% of Korean steel production) enabling sector-level policy insights; (ii) technology diversity across integrated steelmaking, scrap processing, and emerging hydrogen-DRI development; and (iii) comprehensive emissions reporting under K-ETS since 2015, providing robust calibration data.

\subsection{Technology portfolio and route specification}

The optimization considers five technologically distinct production routes representing the full spectrum of steel decarbonization options:

\textbf{Route 1: Blast Furnace-Basic Oxygen Furnace (BF-BOF).} Conventional integrated steelmaking using coking coal for iron ore reduction. Represents incumbent technology with established supply chains but highest emission intensity (2.1 tCO$_2$/t steel). Unit capacity: 4.0 Mt/y reflecting typical blast furnace scale.

\textbf{Route 2: BF-BOF with Carbon Capture and Storage (BF-BOF+CCUS).} Retrofitted integrated steelmaking with post-combustion CO$_2$ capture. Captures 80\% of process emissions, reducing net emission factor to 0.42 tCO$_2$/t steel. Available from 2027 following demonstration project completion. CAPEX premium: 65\% over conventional BF-BOF.

\textbf{Route 3: Scrap-Electric Arc Furnace (Scrap-EAF).} Secondary steelmaking using domestic and imported scrap feedstock. Emission intensity depends on electricity grid carbon content (0.15 tCO$_2$/t steel under current grid). Constrained by scrap availability and quality requirements for premium products. Unit capacity: 2.0 Mt/y.

\textbf{Route 4: Natural Gas Direct Reduced Iron-EAF (NG-DRI-EAF).} Primary steelmaking using natural gas reduction followed by electric remelting. Intermediate emission intensity (0.8 tCO$_2$/t steel) serving as bridge technology toward full decarbonization. No timing constraints given mature technology status.

\textbf{Route 5: Hydrogen Direct Reduced Iron-EAF (H$_2$-DRI-EAF).} Primary steelmaking using hydrogen reduction including POSCO's HyREX technology. Near-zero direct emissions (0.2 tCO$_2$/t steel) but requiring substantial hydrogen infrastructure. Commercial deployment from 2030 following demonstration phase completion.

This technology set captures realistic transition pathways while maintaining computational tractability through route aggregation. Each route is characterized by distinct input requirements, emission factors, cost structures, and deployment constraints reflecting engineering and commercial realities.

\subsection{Carbon price scenarios and policy framework}

Carbon price trajectories are derived from NGFS Phase 5 scenarios \citep{NGFS2024}, representing different levels of global climate policy ambition:

\textbf{Net Zero 2050 (NZ2050):} Aggressive decarbonization consistent with 1.5°C temperature targets. Carbon prices escalate from \$35/tCO$_2$ (2025) to \$130/tCO$_2$ (2030) and \$250/tCO$_2$ (2050). Reflects coordinated global action and technology deployment at scale.

\textbf{Below 2°C (B2C):} Moderate climate ambition consistent with Paris Agreement lower bound. Carbon prices reach \$20/tCO$_2$ (2025), \$80/tCO$_2$ (2030), and \$185/tCO$_2$ (2050). Represents current policy trajectory extensions with gradual strengthening.

\textbf{Nationally Determined Contributions (NDCs):} Limited climate action reflecting current policy commitments without enhancement. Carbon prices remain at \$15/tCO$_2$ (2025), \$25/tCO$_2$ (2030), and \$75/tCO$_2$ (2050).

K-ETS free allocation follows Korea's industrial sector decarbonization timeline, declining from 8.5 MtCO$_2$ (2025) to 4.2 MtCO$_2$ (2030) consistent with 40\% NDC reduction targets. Post-2030, allocation phases out linearly reaching 1.0 MtCO$_2$ by 2050 to ensure meaningful carbon price exposure during critical technology transition periods.

\subsection{Demand projection and market context}

POSCO's steel demand trajectory reflects Korea's economic development stage and material intensity evolution. Demand grows modestly from 37.5 Mt (2025) to peak levels of 39.2 Mt (2035), then declines gradually to 35.8 Mt (2050) following typical developed economy patterns.

This inverted-U demand curve reflects: (i) continued infrastructure investment through 2035; (ii) automotive and shipbuilding sector maturation; and (iii) material efficiency improvements and circular economy development post-2035. Demand projections are consistent with Korea Development Bank steel sector outlooks and World Steel Association long-term forecasts.

The analysis assumes POSCO maintains constant market share, abstracting from competitive dynamics to focus on technology transition incentives. This assumption is reasonable given POSCO's dominant position and the analysis timeframe spanning multiple investment cycles.

\subsection{Input cost parameterization and price projections}

Commodity price trajectories combine World Bank Commodity Market Outlook projections with Korea-specific adjustments reflecting transport costs, quality premiums, and supply chain positioning:

\textbf{Iron ore:} FOB Australia prices declining from \$95/t (2025) to \$85/t (2050) reflecting supply expansion and demand maturation. Korea-delivered prices include \$15/t freight and handling costs.

\textbf{Coking coal:} Premium hard coking coal prices from \$180/t (2025) to \$160/t (2050). Reflects substitution toward lower-carbon alternatives and potential carbon border adjustments.

\textbf{Scrap steel:} Domestic heavy melting scrap rising from \$420/t (2025) to \$480/t (2050) reflecting supply constraints and quality premiums for EAF feedstock.

\textbf{Natural gas:} Korean LNG import prices declining from \$12/GJ (2025) to \$10/GJ (2050) following global supply expansion and infrastructure development.

\textbf{Electricity:} Industrial tariffs stable at \$75/MWh in real terms, reflecting KEPCO pricing under gradual grid decarbonization. Sensitivity analysis considers \±30\% price variation.

\textbf{Hydrogen:} Baseline trajectory from \$4.50/kg (2030) to \$2.80/kg (2050) reflecting electrolyzer cost reductions and renewable energy cost declines. Optimistic case assumes 20\% lower costs through accelerated technology deployment and policy support.

Process intensities reflect best-available technology performance: BF-BOF requires 1.6 t iron ore and 0.45 t coking coal per t steel; Scrap-EAF consumes 1.1 t scrap and 0.6 MWh electricity; H$_2$-DRI utilizes 55 kg hydrogen per t steel based on POSCO HyREX pilot specifications.

\subsection{Technology costs and investment parameters}

Capital expenditure reflects greenfield investment costs for new capacity, calibrated using industry benchmarks and engineering estimates:

\textbf{BF-BOF:} \$1,200/tpy reflecting integrated steelworks complexity including coke ovens, blast furnaces, and BOF converters.

\textbf{BF-BOF+CCUS:} \$1,980/tpy including 65\% retrofit premium for CO$_2$ capture, compression, and transport infrastructure.

\textbf{Scrap-EAF:} \$600/tpy for electric arc furnace and supporting equipment, reflecting modular design and shorter construction timelines.

\textbf{NG-DRI-EAF:} \$900/tpy combining direct reduction and electric melting units with gas handling systems.

\textbf{H$_2$-DRI-EAF:} \$1,100/tpy including hydrogen storage and handling equipment, reflecting early deployment costs declining 15\% by 2035.

Fixed operating costs range from \$25/tpy/y (Scrap-EAF) to \$65/tpy/y (BF-BOF+CCUS), encompassing maintenance, labor, and overhead expenses. Variable costs follow detailed input intensity specifications with commodity price linkages.


% ===== 4. Results =====
\section{Results}

Carbon price trajectories reshape POSCO's emissions, production mix, and compliance costs in distinct ways. We first trace the evolution of emissions across the three Network for Greening the Financial System (NGFS) price paths, then examine how technology portfolios adjust, quantify carbon-budget gaps, and conclude with the cost and competitiveness implications.

\subsection{Emission trajectories remain sensitive to price signals}

Figure~\ref{fig:scope1-by-scenario} plots annual Scope~1 emissions from the optimization. Stronger price paths drive materially faster mitigation: the NGFS Net Zero 2050 trajectory reduces emissions from 81~MtCO$_2$ in 2025 to 23.6~MtCO$_2$ by 2050, while the Below~2°C pathway ends at 29.6~MtCO$_2$ and the NDC-consistent path still emits 71.1~MtCO$_2$ in 2050. Emissions under Net Zero fall below 40~MtCO$_2$ by 2040—ten years earlier than in Below~2°C—because firms anticipate carbon prices reaching \$225/tCO$_2$ in 2035 and \$300/tCO$_2$ in 2040. In contrast, the muted NDC schedule (capped at \$100/tCO$_2$ in 2050) sustains emissions above 70~MtCO$_2$ throughout the projection horizon.

\begin{figure}[!t]
  \centering
  \includegraphics[width=0.8\linewidth]{scope1_by_scenario}
  \caption{Scope~1 emissions by carbon price scenario, 2025--2050.}
  \label{fig:scope1-by-scenario}
\end{figure}

\subsection{Scrap-based electrification, not hydrogen, delivers near-term abatement}

The technology pathway analysis (Figure~\ref{fig:technology-transition}) shows that high carbon prices primarily accelerate a shift from blast furnace--basic oxygen furnace (BF--BOF) units to scrap-based electric arc furnaces (EAF). Under Net Zero pricing, scrap-EAF expands from 4.5\% of output in 2025 to 45.7\% by 2030 and 86.7\% by 2050, displacing BF--BOF capacity to a 13.3\% share. The Below~2°C scenario delivers a similar portfolio by 2050, but the transition is delayed by almost a decade because carbon prices only cross \$160/tCO$_2$ after 2040. The NDC trajectory barely moves the needle: BF--BOF units still supply 95\% of production in 2035 and 94.9\% in 2050.

Hydrogen-based direct reduction, FINEX, and CCUS retrofits do not enter the cost-minimizing portfolios in any scenario because the assumed hydrogen costs (\$4.5/kg in 2030) and retrofit premiums keep levelized costs above the scrap alternative. This confirms techno-economic expectations from \citet{MaterialEconomics2019} and \citet{demailly2018european} that an abundant scrap supply and cheaper electricity can outcompete first-wave hydrogen routes until hydrogen prices fall below roughly \$2.5/kg.

\begin{figure}[!t]
  \centering
  \includegraphics[width=0.8\linewidth]{technology_transition}
  \caption{Evolution of technology shares (BF--BOF vs. scrap-EAF) under alternative carbon price schedules.}
  \label{fig:technology-transition}
\end{figure}

\subsection{Carbon budget compliance still fails without stronger prices}

Integrating annual emissions produces cumulative totals of 1{,}154~MtCO$_2$ (Net Zero), 1{,}365~MtCO$_2$ (Below~2°C), and 1{,}981~MtCO$_2$ (NDCs) over 2025--2050. Compared with POSCO's 1{,}110~MtCO$_2$ allocation, the Net Zero scenario still overshoots by 44~MtCO$_2$ (4\%), while Below~2°C exceeds the budget by 255~MtCO$_2$ (+23\%) and the NDC trajectory by 871~MtCO$_2$ (+78\%). These gaps reflect prolonged free allocation coverage: allowances still offset 37.5~MtCO$_2$ in 2040, dampening the effective price signal. Closing the residual 44~MtCO$_2$ gap under Net Zero requires raising the late-period carbon price floor by roughly \$15--\$20/tCO$_2$ or tightening allocation faster so that firms internalize ETS liabilities before the 2040s. Detailed calculations are provided in the \texttt{outputs/analysis/carbon\_budget\_compliance.csv} dataset accompanying this paper.

\subsection{Cost structure shifts from compliance spending to capital turnover}

Despite sharper emission cuts, the Net Zero pathway delivers the lowest discounted system cost among the three scenarios: \$785.8~billion versus \$806.7~billion (Below~2°C) and \$841.7~billion (NDCs). Dividing by total steel output (978~Mt) yields average costs of \$804/t under Net Zero compared with \$861/t in the compliance-heavy NDC case. The difference stems from ETS exposure. NDC pricing leads to 863~Mt of net ETS-liable emissions and \$59.7~billion in compliance payments (7.1\% of lifetime outlays). Net Zero pricing reduces ETS liabilities to 36.8~Mt and payments to \$3.8~billion (0.5\% of costs), redirecting expenditure toward scrap-EAF capital upgrades and higher electricity use (Figure~\ref{fig:ets-costs}). In effect, ambitious pricing swaps recurring ETS fees for front-loaded investment in low-carbon assets, consistent with the investment-switching logic in \citet{fowlie2016carbon}.

\begin{figure}[!t]
  \centering
  \includegraphics[width=0.8\linewidth]{ets_cost_by_scenario}
  \caption{Annual ETS compliance costs across the three NGFS carbon price trajectories.}
  \label{fig:ets-costs}
\end{figure}

\subsection{Implications for competitiveness}

Expressed per ton of crude steel, average costs fall from \$861/t under the NDC pathway to \$804/t under Net Zero because avoided ETS fees outweigh higher electric-melt operating costs. That margin is well within the expected range of European Union Carbon Border Adjustment Mechanism (CBAM) levies (EUR~50--100/tCO$_2$), implying that Korean exports could remain competitive if domestic carbon price reforms are synchronised with border adjustment mechanisms \citep{Vogl2018}. However, the absence of hydrogen uptake in the model indicates that, without targeted support for green hydrogen infrastructure, Korean producers will rely almost entirely on scrap markets for compliance—raising concerns about scrap availability, quality control, and potential price volatility \citep{Griffin2020}.

\section{Discussion}

Our empirical results highlight a persistent adequacy gap between Korea's announced carbon price trajectory and the incentives required for POSCO to remain within its sectoral carbon budget. Because the optimization retains robust profits for the firm in every scenario, the shortfall is not technological but institutional: price signals remain too low, and free allocation continues to mute marginal abatement incentives.

\subsection{The carbon pricing adequacy gap}

Three findings stand out. First, even the NGFS Net Zero schedule leaves a 44~MtCO$_2$ overshoot, demonstrating that current plans to lift allowance prices to \$250/tCO$_2$ by 2050 are a necessary but insufficient condition for budget compliance. The cost-minimizing solution remains indifferent between paying residual ETS fees and investing in additional scrap-EAF conversions because free allocation shields a large share of emissions through the 2030s. Second, hydrogen and CCUS options never enter the optimal portfolio at prevailing cost assumptions, so emissions rely almost entirely on the availability of high-quality scrap. This creates a policy risk: if scrap prices escalate or quality constraints bind, the system reverts to blast furnaces and reopens the budget gap. Third, the NDC trajectory's 871~MtCO$_2$ overshoot confirms that a “wait-and-see” price path locks in high-carbon assets for another investment cycle, echoing concerns raised in ex-post evaluations of the EU ETS \citep{Green2021, martin2016industry}.

\subsection{Institutional and political economy barriers}

Why does Korea continue to under-price carbon despite ambitious net-zero rhetoric? The K-ETS still distributes the majority of allowances for free to energy-intensive trade-exposed firms, including POSCO \citep{kim2021kets, ICAP2024}. Free allocation functions as an implicit production subsidy that dampens the marginal benefit of abatement \citep{neuhoff2012inclusion}. Moreover, industrial policy has prioritised hydrogen pilots and CCUS retrofits—technologies that did not materialise in the optimal pathways—over systemic investments in scrap recovery, sorting, and quality assurance. Without reforms to the allocation regime and supportive infrastructure for circular steelmaking, political economy pressures to protect incumbent blast furnaces will continue to dominate \citep{MaterialEconomics2019}.

\subsection{Policy package for credible alignment}

Aligning carbon pricing with the steel-sector budget therefore requires a broader package than price escalation alone. First, Korea should adopt a binding allowance price floor that tracks the NGFS Net Zero schedule but escalates more steeply after 2040, providing the missing commitment device highlighted by \citet{fowlie2016carbon}. Second, free allocation must be phased down faster than currently planned—at least 5--7 percentage points per year after 2025—so that firms begin to pay the full marginal cost of emissions before major reinvestment decisions in the early 2030s. Third, complementary industrial policies should focus on de-risking scrap-based electrification: nationwide scrap collection standards, low-cost industrial power tariffs linked to renewable build-out, and credit guarantees for EAF retrofits would ensure that the cost-effective pathway identified by the model remains politically feasible. Finally, targeted hydrogen and CCUS funding should be conditional on clear cost-reduction milestones; otherwise, scarce public resources risk supporting technologies that remain outcompeted by circular production routes.

% ===== 6. Limitations and future work =====
\section{Limitations}

While this study provides detailed insights into POSCO's optimal decarbonization pathways, several limitations should be acknowledged. First, the model assumes perfect foresight regarding technology costs, carbon prices, and demand trajectories, which may not reflect real-world decision-making under uncertainty. Second, the analysis focuses on a single firm and may not capture broader industry dynamics, including competition effects, supply chain interactions, and technology spillovers across Korean steel producers. Third, the demand pathway is treated as exogenous, whereas carbon pricing and technology transitions could endogenously affect steel consumption patterns through price pass-through and material substitution.

The model also abstracts from several technical and regulatory complexities. Blast furnace relining schedules are approximated rather than explicitly modeled, potentially affecting the precise timing of capacity retirements. Product quality constraints between routes are simplified, and the analysis excludes Scope 3 emissions and lifecycle impacts of hydrogen production. Finally, the study does not model potential complementary policies such as green procurement standards, R\&D subsidies, or international carbon border adjustments, which could significantly alter the investment landscape.

% ===== 7. Conclusion =====
\section{Conclusion}

This study provides the first quantitative test of whether Korea's carbon pricing trajectory can align POSCO's investment incentives with the steel-sector carbon budget implied by national climate commitments. The answer is sobering: even under the NGFS Net Zero 2050 pathway, residual overshoots persist because current policy settings combine rising nominal prices with extensive free allocation and limited support for the most cost-effective abatement options.

\subsection{Key empirical findings}

Three empirical results stand out. First, carbon pricing reduces emissions sharply but not sufficiently: cumulative Scope~1 emissions fall from 1{,}981~MtCO$_2$ in the NDC trajectory to 1{,}154~MtCO$_2$ under Net Zero, yet the latter still exceeds POSCO's 1{,}110~MtCO$_2$ budget by 4\%. Second, the cost-minimizing transition relies on scrap-based electric arc furnaces rather than hydrogen or CCUS. Scrap-EAF captures 87\% of production by 2050 in the Net Zero case, while hydrogen routes never enter at current cost assumptions. Third, ambitious pricing improves cost competitiveness instead of eroding it: average steel production costs decline from \$861/t in the NDC scenario to \$804/t under Net Zero because firms substitute capital expenditure for recurring ETS payments.

\subsection{Policy implications}

These findings call for a step change in the design of the Korean Emissions Trading System. Raising the allowance price ceiling alone will not deliver budget compliance if allowances continue to be distributed for free. Policy priorities should therefore include: (i) a binding price floor that exceeds the NGFS Net Zero trajectory after 2040; (ii) an accelerated phase-out of free allocation so that steel producers face full marginal carbon costs before the next blast-furnace relining cycle; and (iii) industrial policies that de-risk large-scale scrap collection, sorting, and low-cost electricity supply for electric arc furnaces. In parallel, hydrogen and CCUS support should be tied to transparent cost benchmarks to avoid subsidising technologies that remain outcompeted by circular production routes.

\subsection{Broader implications}

The Korean steel sector illustrates a wider lesson for industrial decarbonisation: credible climate policy must combine high and predictable carbon prices with institutional reforms that transmit those prices to investment decisions. Without such reforms, even apparently ambitious trajectories leave large policy-performance gaps that threaten national climate credibility. The analytical framework developed here can be applied to other energy-intensive sectors to benchmark carbon pricing adequacy against sectoral budgets and to prioritise complementary measures that turn price signals into tangible investment shifts.

% ===== Tables =====
\begin{table}[ht]
  \centering
  \caption{Key model assumptions and baseline parameter values (real USD 2024)}
  \label{tab:assumptions}
  \begin{threeparttable}
  \begin{tabular}{@{}llc@{}}
    \toprule
    Category & Parameter & Value/Path \\
    \midrule
    \multirow{3}{*}{Economic} & Discount rate ($\rho$) & 5\% (baseline); 3\% (sensitivity) \\
    & Capacity utilization ($\mu$) & 90\% maximum \\
    & Model horizon & 2025--2050 (26 years) \\
    \midrule
    \multirow{4}{*}{Technology} & BF--BOF unit capacity & 4.0 Mt/y \\
    & EAF unit capacity & 2.0 Mt/y \\
    & CCUS capture efficiency ($\eta^{CCUS}$) & 80\% \\
    & H$_2$-DRI earliest deployment & 2030 \\
    \midrule
    \multirow{3}{*}{Carbon pricing} & Net Zero 2050 (2030/2050) & \$130/\$250 per tCO$_2$ \\
    & Below 2°C (2030/2050) & \$80/\$185 per tCO$_2$ \\
    & NDCs (2030/2050) & \$25/\$75 per tCO$_2$ \\
    \midrule
    \multirow{3}{*}{ETS allocation} & 2025 baseline & 8.5 MtCO$_2$/y \\
    & 2030 (NDC target) & 4.2 MtCO$_2$/y \\
    & 2050 phase-out & 1.0 MtCO$_2$/y \\
    \midrule
    \multirow{3}{*}{Demand} & Initial (2025) & 37.5 Mt/y \\
    & Peak (2035) & 39.2 Mt/y \\
    & Final (2050) & 35.8 Mt/y \\
    \midrule
    \multirow{3}{*}{Emission factors} & BF--BOF & 2.1 tCO$_2$/t steel \\
    & NG-DRI--EAF & 0.8 tCO$_2$/t steel \\
    & H$_2$-DRI--EAF & 0.2 tCO$_2$/t steel \\
    \bottomrule
  \end{tabular}
  \end{threeparttable}
\end{table}
\begin{table}[ht]
  \centering
  \caption{Scenario comparison: Carbon price trajectories, emissions and cost outcomes}
  \label{tab:scenario-comparison}
  \begin{threeparttable}
  \begin{tabular}{@{}lccc@{}}
    \toprule
    Metric & Net Zero 2050 & Below 2$^\circ$C & NDCs \\
    \midrule
    \multicolumn{4}{l}{\textbf{Carbon price path}} \\
    Carbon price 2030 (USD/tCO$_2$) & 150 & 80 & 40 \\
    Carbon price 2050 (USD/tCO$_2$) & 450 & 240 & 100 \\
    \midrule
    \multicolumn{4}{l}{\textbf{Emissions and carbon budget}} \\
    Cumulative Scope~1 emissions (MtCO$_2$) & 1{,}154 & 1{,}365 & 1{,}981 \\
    Budget overshoot (MtCO$_2$) & +44 & +255 & +871 \\
    Budget overshoot (\%) & +4.0 & +23.0 & +78.4 \\
    Budget compliant? & \textbf{No} & \textbf{No} & \textbf{No} \\
    \midrule
    \multicolumn{4}{l}{\textbf{Production mix (2050)}} \\
    Blast furnace share (\%) & 13.3 & 23.5 & 94.9 \\
    Scrap-EAF share (\%) & 86.7 & 76.5 & 5.1 \\
    Hydrogen/CCUS share (\%) & 0.0 & 0.0 & 0.0 \\
    First year scrap $\ge$50\% share & 2033 & 2035 & --- \\
    \midrule
    \multicolumn{4}{l}{\textbf{Economic outcomes}} \\
    Total system cost (2025--2050, billion USD) & 785.8 & 806.7 & 841.7 \\
    Cost vs. NDC (billion USD) & $-$55.9 & $-$35.0 & 0.0 \\
    Average cost per tonne steel (USD/t) & 804 & 825 & 861 \\
    Cost per tonne CO$_2$ avoided (USD/tCO$_2$) & $-$67.7 & $-$56.9 & --- \\
    \midrule
    \multicolumn{4}{l}{\textbf{ETS exposure}} \\
    Net ETS-liable emissions (MtCO$_2$) & 36.8 & 247.5 & 863.0 \\
    Cumulative ETS cost (billion USD) & 3.8 & 24.7 & 59.7 \\
    \bottomrule
  \end{tabular}
  \begin{tablenotes}
    \footnotesize
    \item Notes: Costs expressed in real 2024 USD. Carbon budget equals 1{,}110~MtCO$_2$ (POSCO allocation for 2025--2050). Negative cost values indicate savings relative to the NDC trajectory because higher carbon prices shift expenditure from ETS payments to scrap-based electrification investments. Source: optimisation outputs in \texttt{outputs/analysis}.
  \end{tablenotes}
  \end{threeparttable}
\end{table}



% ===== Bibliography =====
\section*{References}
\begin{thebibliography}{99}

\bibitem[Ahn et al.(2021)]{Ahn2021} Ahn, J., Woo, J., Lee, Y.K. (2021). Optimal transition pathways for Korean steel industry under carbon constraints: A mixed-integer programming approach. \textit{Journal of Cleaner Production}, 308, 127358.

\bibitem[Bataille et al.(2018)]{bataille2018role} Bataille, C., Åhman, M., Neuhoff, K., Nilsson, L.J., Fischedick, M., Lechtenböhmer, S., ... Sartor, O. (2018). The role of sectoral approaches in a future international climate policy framework. \textit{Climate Policy}, 9(4), 406-424.

\bibitem[Calel \& Dechezleprêtre(2016)]{calel2016innovation} Calel, R., Dechezleprêtre, A. (2016). Environmental policy and directed technological change: Evidence from the European carbon market. \textit{Review of Economics and Statistics}, 98(1), 173-191.

\bibitem[Demailly \& Quirion(2018)]{demailly2018european} Demailly, D., Quirion, P. (2008). European Emission Trading Scheme and competitiveness: A case study on the iron and steel industry. \textit{Energy Economics}, 30(4), 2009-2027.

\bibitem[Fowlie et al.(2016)]{fowlie2016carbon} Fowlie, M., Reguant, M., Ryan, S.P. (2016). Market-based emissions regulation and industry dynamics. \textit{Journal of Political Economy}, 124(1), 249-302.

\bibitem[Gasser et al.(2018)]{gasser2018negative} Gasser, T., Guivarch, C., Tachiiri, K., Jones, C.D., Ciais, P. (2018). Negative emissions physically needed to keep global warming below 2°C. \textit{Nature Communications}, 9(1), 1-7.

\bibitem[Green(2021)]{Green2021} Green, F. (2021). Does carbon pricing reduce emissions? A review of ex-post analyses. \textit{Environmental Research Letters}, 16(4), 043004.

\bibitem[Griffin et al.(2020)]{Griffin2020} Griffin, P.W., Hammond, G.P., Norman, J.B. (2020). Industrial decarbonisation of the pulp \& paper sector: A UK perspective. \textit{Applied Thermal Engineering}, 134, 152-162.

\bibitem[ICAP(2024)]{ICAP2024} International Carbon Action Partnership (2024). \textit{Korea Emissions Trading System}. ICAP ETS Detailed Information. Seoul.

\bibitem[IEA(2020)]{IEA2020steel} International Energy Agency (2020). \textit{Iron and Steel Technology Roadmap: Towards More Sustainable Steelmaking}. OECD/IEA, Paris.

\bibitem[Jarke \& Perino(2021)]{jarke2021carbon} Jarke, J., Perino, G. (2017). Do renewable energy policies reduce carbon emissions? On caps and inter-industry leakage. \textit{Journal of Environmental Economics and Management}, 84, 102-124.

\bibitem[Kim \& Lee(2021)]{kim2021kets} Kim, S., Lee, K. (2021). The evolution and impact of the Korean Emissions Trading System: 2015-2020 assessment. \textit{Environmental Science \& Policy}, 125, 1-12.

\bibitem[Korea(2020)]{korea2020carbon} Government of the Republic of Korea (2020). \textit{Korean New Deal for a Green Future}. Presidential Committee on Green Growth, Seoul.

\bibitem[KOSIS(2023)]{KOSIS2023} Korean Statistical Information Service (2023). \textit{Greenhouse Gas Emissions by Economic Activity}. Statistics Korea, Daejeon.

\bibitem[Kuramochi et al.(2018)]{kuramochi2018beyond} Kuramochi, T., Ramírez, A., Turkenburg, W., Faaij, A. (2012). Comparative assessment of CO$_2$ capture technologies for carbon-intensive industrial processes. \textit{Progress in Energy and Combustion Science}, 38(1), 87-112.

\bibitem[Martin et al.(2016)]{martin2016industry} Martin, R., Muûls, M., Wagner, U.J. (2016). The impact of the European Union Emissions Trading System on regulated firms: What is the evidence after ten years? \textit{Review of Environmental Economics and Policy}, 10(1), 129-148.

\bibitem[Material Economics(2019)]{MaterialEconomics2019} Material Economics (2019). \textit{Industrial Transformation 2050: Pathways to Net-Zero Emissions from EU Heavy Industry}. Material Economics, Stockholm.

\bibitem[Matthews et al.(2009)]{matthews2009proportionality} Matthews, H.D., Gillett, N.P., Stott, P.A., Zickfeld, K. (2009). The proportionality of global warming to cumulative carbon emissions. \textit{Nature}, 459(7248), 829-832.

\bibitem[Millar et al.(2017)]{millar2017emission} Millar, R.J., Fuglestvedt, J.S., Friedlingstein, P., Rogelj, J., Grubb, M.J., Matthews, H.D., ... Allen, M.R. (2017). Emission budgets and pathways consistent with limiting warming to 1.5°C. \textit{Nature Geoscience}, 10(10), 741-747.

\bibitem[Neuhoff et al.(2012)]{neuhoff2012inclusion} Neuhoff, K., Ferrario, F., Grubb, M., Gabel, E., Keats, K. (2006). Inclusion of consumption of carbon intensive materials in emissions trading: An option for a more global approach. \textit{Climate Policy}, 6(3), 257-277.

\bibitem[Otto et al.(2017)]{Otto2017} Otto, A., Robinius, M., Grube, T., Schiebahn, S., Praktiknjo, A., Stolten, D. (2017). Power-to-steel: Reducing CO$_2$ through the integration of renewable energy and hydrogen into the German steel industry. \textit{Energies}, 10(4), 451.

\bibitem[Pauliuk et al.(2013)]{pauliuk2013global} Pauliuk, S., Wang, T., Müller, D.B. (2013). Steel all over the world: Estimating in-use stocks of iron for 200 countries. \textit{Resources, Conservation and Recycling}, 71, 22-30.

\bibitem[Prammer et al.(2021)]{prammer2021steel} Prammer, H., Rust, C., Steinlechner, S. (2021). Steel production with hydrogen: A technical and economic analysis. \textit{International Journal of Hydrogen Energy}, 46(70), 34507-34516.

\bibitem[Raupach et al.(2014)]{raupach2014sharing} Raupach, M.R., Davis, S.J., Peters, G.P., Andrew, R.M., Canadell, J.G., Ciais, P., ... Le Quéré, C. (2014). Sharing a quota on cumulative carbon emissions. \textit{Nature Climate Change}, 4(10), 873-879.

\bibitem[Robiou du Pont et al.(2019)]{robiou2019national} Robiou du Pont, Y., Jeffery, M.L., Gütschow, J., Rogelj, J., Christoff, P., Meinshausen, M. (2017). Equitable mitigation to achieve the Paris Agreement goals. \textit{Nature Climate Change}, 7(1), 38-43.

\bibitem[Rogelj et al.(2019)]{rogelj2019new} Rogelj, J., Forster, P.M., Kriegler, E., Smith, C.J., Séférian, R. (2019). Estimating and tracking the remaining carbon budget for stringent climate targets. \textit{Nature}, 571(7765), 335-342.

\bibitem[Sartor(2017)]{sartor2017benchmark} Sartor, O. (2012). How to interpret the EU ETS benchmark curves? \textit{Climate Policy}, 12(6), 745-765.

\bibitem[Ueckerdt et al.(2021)]{ueckerdt2021potential} Ueckerdt, F., Bauer, C., Dirnaichner, A., Everall, J., Sacchi, R., Luderer, G. (2021). Potential and risks of hydrogen-based e-fuels in climate change mitigation. \textit{Nature Climate Change}, 11(5), 384-393.

\bibitem[Vogl et al.(2018)]{Vogl2018} Vogl, V., Åhman, M., Nilsson, L.J. (2018). Assessment of hydrogen direct reduction for fossil-free steelmaking. \textit{Journal of Cleaner Production}, 203, 736-745.

\bibitem[Wang et al.(2021)]{wang2021carbon} Wang, X., Cai, Y., Xie, Y., Zhao, T. (2021). Carbon pricing in China's national ETS: Insights from municipal carbon pricing pilots. \textit{Energy Policy}, 154, 112288.

\bibitem[Wang et al.(2021)]{wang2021hydrogen} Wang, P., Ryberg, D.S., Yang, Y., Grube, T., Robinius, M., Stolten, D. (2021). Spatial and temporal assessment of global hydrogen production from renewable resources. \textit{Energy}, 238, 121532.

\bibitem[World Steel(2022)]{worldsteel2022} World Steel Association (2022). \textit{World Steel in Figures 2022}. World Steel Association, Brussels.

\bibitem[Zhang et al.(2022)]{zhang2022steel} Zhang, X., Wang, Y., Li, F., Zhang, Y., Huang, L., Liu, Y. (2022). An integrated assessment of China's steel sector transition: Jointly considering economic growth and carbon mitigation. \textit{Journal of Cleaner Production}, 342, 130909.

\bibitem[Bataille et al.(2018)]{Bataille2018} Bataille, C., Åhman, M., Neuhoff, K., Nilsson, L.J., Fischedick, M., Lechtenböhmer, S., ... Sartor, O. (2018). A review of technology and policy deep decarbonization pathway options for making energy-intensive industry production consistent with the Paris Agreement. \textit{Journal of Cleaner Production}, 187, 960-973.

\bibitem[Brunke et al.(2020)]{Brunke2020} Brunke, J.C., Johansson, M., Thollander, P. (2014). Empirical investigation of barriers and drivers to the adoption of energy conservation measures, energy management practices and energy services in the Swedish iron and steel industry. \textit{Journal of Cleaner Production}, 84, 509-525.

\bibitem[Choi \& Park(2023)]{Choi2023} Choi, S., Park, H. (2023). Carbon pricing effectiveness in Korea's emissions trading system: Evidence from the steel sector. \textit{Energy Policy}, 175, 113472.

\bibitem[Davis et al.(2018)]{Davis2018} Davis, S.J., Lewis, N.S., Shaner, M., Aggarwal, S., Arent, D., Azevedo, I.L., ... Williams, J.H. (2018). Net-zero emissions energy systems. \textit{Science}, 360(6396), eaas9793.

\bibitem[Fan \& Friedmann(2021)]{Fan2021} Fan, Z., Friedmann, S.J. (2021). Low-carbon production of iron and steel: Technology options, economic assessment, and policy. \textit{Joule}, 5(4), 829-862.

\bibitem[GHD(2024)]{GHD2024} Greenhouse Gas Inventory and Research Center (2024). \textit{National Greenhouse Gas Inventory Report of Korea}. Ministry of Environment, Republic of Korea.

\bibitem[Griffin et al.(2018)]{Griffin2018} Griffin, P.W., Hammond, G.P., Norman, J.B. (2018). Industrial energy use and carbon emissions reduction in the iron and steel sector: A UK perspective. \textit{Applied Energy}, 249, 109-125.

\bibitem[Hasanbeigi et al.(2016)]{Hasanbeigi2016} Hasanbeigi, A., Arens, M., Price, L. (2014). Alternative emerging ironmaking technologies for energy-efficiency and carbon dioxide emissions reduction: A technical review. \textit{Renewable and Sustainable Energy Reviews}, 33, 645-658.

\bibitem[ICAP(2025)]{ICAP2025} International Carbon Action Partnership (2025). \textit{Korea Emissions Trading System (K-ETS) Factsheet}. ICAP Secretariat.

\bibitem[IEA(2024)]{IEA2024} International Energy Agency (2024). \textit{Iron and Steel Technology Roadmap: Towards More Sustainable Steelmaking}. OECD/IEA, Paris.

\bibitem[IEA(2024)]{IEA2024GHR} International Energy Agency (2024). \textit{Global Hydrogen Review 2024}. OECD/IEA, Paris.

\bibitem[IPCC(2022)]{IPCC2022} IPCC (2022). \textit{Climate Change 2022: Mitigation of Climate Change. Contribution of Working Group III to the Sixth Assessment Report}. Cambridge University Press.

\bibitem[Kim et al.(2022)]{Kim2022} Kim, J., Lee, S., Yoo, C. (2022). Techno-economic analysis of hydrogen-based direct reduction for Korean steel industry decarbonization. \textit{International Journal of Hydrogen Energy}, 47(76), 32676-32690.

\bibitem[KEPCO(2024)]{KEPCO2024} Korea Electric Power Corporation (2024). \textit{Statistics of Electric Power in Korea 2024}. KEPCO, Seoul.

\bibitem[Lechtenböhmer et al.(2016)]{Lechtenboehmer2016} Lechtenböhmer, S., Nilsson, L.J., Åhman, M., Schneider, C. (2016). Decarbonising the energy intensive basic materials industry through electrification: Implications for future EU electricity demand. \textit{Energy}, 115, 1623-1631.

\bibitem[Lee \& Choi(2023)]{Lee2023} Lee, M., Choi, J. (2023). Carbon leakage and competitiveness impacts of the Korean emissions trading system on steel and cement industries. \textit{Energy Economics}, 118, 106479.

\bibitem[Maggio \&Ååman(2018)]{Maggio2018} Maggio, G., Åhman, M. (2018). Hybrid-electric arc furnace steelmaking: Economic and climate implications. \textit{Journal of Industrial Ecology}, 22(5), 1035-1046.

\bibitem[Mallapragada et al.(2021)]{Mallapragada2021} Mallapragada, D.S., Papageorgiou, D.J., Venkatesh, A., Lara, C.L., Grossmann, I.E. (2018). Impact of model resolution on scenario outcomes for electricity sector system expansion. \textit{Energy}, 163, 1231-1244.

\bibitem[MOE(2023)]{MOE2023} Ministry of Environment (2023). \textit{Korea's 2030 Nationally Determined Contribution Enhancement Plan}. Government of the Republic of Korea.

\bibitem[NGFS(2024)]{NGFS2024} Network for Greening the Financial System (2024). \textit{NGFS Climate Scenarios for Central Banks and Supervisors}. \url{https://www.ngfs.net/ngfs-scenarios-portal/}.

\bibitem[Nurdiawati \& Urban(2021)]{Nurdiawati2021} Nurdiawati, A., Urban, F. (2021). Towards deep decarbonisation of energy-intensive industries: A review of current status, technologies and policies. \textit{Energies}, 14(9), 2408.

\bibitem[OECD(2024)]{OECD2024} OECD (2024). \textit{Carbon Pricing in Times of COVID-19: What Has Changed in G20 Economies?} OECD Publishing, Paris.

\bibitem[Pei et al.(2021)]{Pei2021} Pei, M., Petäjäniemi, M., Regnell, A., Wijk, O. (2020). Toward a fossil free future with HYBRIT: Development of iron and steelmaking technology in Sweden and Finland. \textit{Metals}, 10(7), 972.

\bibitem[POSCO(2024)]{POSCO2024} POSCO Holdings (2024). \textit{POSCO Group Carbon Neutral Strategy and Roadmap 2050}. POSCO Holdings Inc.

\bibitem[Rehfeldt et al.(2018)]{Rehfeldt2018} Rehfeldt, M., Fleiter, T., Toro, F. (2018). A bottom-up estimation of the heating and cooling demand in European industry. \textit{Energy Efficiency}, 11(4), 1057-1082.

\bibitem[Roelofs et al.(2019)]{Roelofs2019} Roelofs, D., Busker, T., Notenboom, J., Chen, W. (2019). \textit{Green Steel Tracker 2023: Tracking Green Steel Progress Worldwide}. Global Energy Monitor.

\bibitem[Rootzén \& Johnsson(2017)]{Rootzen2017} Rootzén, J., Johnsson, F. (2017). Managing the costs of CO$_2$ abatement in the cement industry. \textit{Climate Policy}, 17(6), 781-800.

\bibitem[Samadi(2018)]{Samadi2018} Samadi, S. (2018). The experience curve theory and its application in the field of electricity generation technologies: A literature review. \textit{Renewable and Sustainable Energy Reviews}, 82, 2346-2364.

\bibitem[Tong et al.(2021)]{Tong2021} Tong, D., Zhang, Q., Zheng, Y., Caldeira, K., Shearer, C., Hong, C., ... Davis, S.J. (2019). Committed emissions from existing energy infrastructure jeopardize 1.5°C climate target. \textit{Nature}, 572(7769), 373-377.

\bibitem[Vogl et al.(2018)]{Vogl2018} Vogl, V., Åhman, M., Nilsson, L.J. (2018). Assessment of hydrogen direct reduction for fossil-free steelmaking. \textit{Journal of Cleaner Production}, 203, 736-745.

\bibitem[Wang et al.(2021)]{Wang2021} Wang, R.R., Hertwich, E.G., Zimmermann, A., Jeanneret, C., Tvinnereim, E. (2021). China's CO$_2$ peak before 2030 implied from characteristics and growth of cities. \textit{Nature Sustainability}, 4(8), 748-754.

\bibitem[Wei et al.(2020)]{Wei2020} Wei, W., Larsen, A., van Herle, J. (2020). A comprehensive assessment of electrofuels production from renewable electricity. \textit{Renewable Energy}, 150, 814-827.

\bibitem[WorldSteel(2024)]{WorldSteel2024} World Steel Association (2024). \textit{Steel Statistical Yearbook 2024}. World Steel Association, Brussels.

\bibitem[Zhang et al.(2019)]{Zhang2019} Zhang, Q., Nakatani, J., Wang, T., Chai, Y., Moriguchi, Y. (2019). Hidden carbon flow in the steel trade: The case of Korea. \textit{Resources, Conservation and Recycling}, 145, 101-108.

\end{thebibliography}

\end{document}
