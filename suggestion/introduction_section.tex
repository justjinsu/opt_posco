\section{Introduction}

Can market-based climate policies deliver the massive industrial transformation required for net-zero emissions? This question has become increasingly urgent as governments worldwide deploy carbon pricing to decarbonize energy-intensive industries. Despite the proliferation of emissions trading systems covering millions of firms across dozens of countries, we still lack robust empirical evidence on whether carbon prices can actually drive the deep technological shifts required to meet climate targets \citep{Green2021}. This paper addresses that gap by testing whether Korea's carbon pricing system can align steel industry investments with national climate commitments.

The stakes are particularly high in steel production, which accounts for roughly 7\% of global CO$_2$ emissions \citep{worldsteel2022}. Korea's steel sector offers an especially revealing test case. As the world's sixth-largest producer, Korea's industry is dominated by a single firm—POSCO—whose emissions alone represent 10\% of the nation's greenhouse gas inventory \citep{KOSIS2023}. To put this in perspective, POSCO's annual emissions exceed those of entire countries like Belgium or Chile. This industrial concentration creates both an opportunity and a challenge: if carbon pricing cannot drive transformation in such a focused, high-emission sector with clear policy leverage, its prospects for broader industrial decarbonization look questionable.

What makes steel particularly challenging is the sheer scale of technological disruption required. Moving from today's carbon-intensive blast furnace routes to low-carbon alternatives—whether hydrogen-based direct reduction, carbon capture and storage, or electric arc furnaces powered by clean electricity—demands wholesale replacement of production infrastructure \citep{IEA2020steel}. Individual plant investments often exceed \$2 billion, and once built, these facilities operate for 25 to 40 years \citep{MaterialEconomics2019}. The lumpy and irreversible nature of these capital decisions creates powerful inertia that carbon price signals must overcome. A steel company cannot gradually shift its production mix the way a power generator can dispatch different plants; it must commit to discrete, long-lived technology choices that lock in emission trajectories for decades.

Korea's policy landscape provides a compelling context for examining these dynamics. The Korean Emissions Trading System (K-ETS), launched in 2015, now covers 70\% of national emissions including the steel sector \citep{kim2021kets}. However, the system has historically operated with generous free allocation—POSCO received allowances covering approximately 95\% of its emissions in recent years \citep{ICAP2024}—effectively insulating the sector from carbon costs. This protection is scheduled to decline as Korea pursues its 2030 nationally determined contribution (a 40\% reduction from 2018 levels) and 2050 carbon neutrality goal, gradually exposing steel producers to meaningful carbon price signals. The question is whether this evolving price trajectory will prove sufficient to drive the necessary investment response.

To frame this question more precisely, we need to establish what "sufficient" means in climate terms. Korea's carbon neutrality framework implies a finite carbon budget for each economic sector \citep{korea2020carbon}. Drawing on Korea's national climate commitments and standard burden-sharing principles, we estimate that the steel sector can emit approximately 1,850 MtCO$_2$ between 2025 and 2050 while remaining consistent with national targets. Given POSCO's roughly 60\% share of domestic steel production, this translates to a firm-level budget of about 1,110 MtCO$_2$ over the period. The fundamental test of carbon pricing adequacy, then, is whether profit-maximizing corporate investment decisions—responding to planned carbon price trajectories—will stay within this budget envelope. Section 3 provides full details on our budget derivation methodology.

We examine this question using a mixed-integer linear programming model that optimizes POSCO's technology portfolio under three carbon price scenarios drawn from the Network for Greening the Financial System (NGFS) \citep{NGFS2024}. These globally consistent scenarios—Net Zero 2050, Below 2$^\circ$C, and NDCs—reflect different levels of climate ambition and are widely used by central banks and financial regulators for climate risk assessment. Our model minimizes the net present value of total system costs (capital expenditure, operating costs, and carbon compliance costs) subject to realistic constraints on technology deployment, feedstock availability, and product quality. Unlike previous steel optimization studies that treat capacity additions as continuous variables, we explicitly model the discrete, lumpy nature of blast furnace investment cycles. By comparing the resulting emission pathways against our derived carbon budget, we can assess whether current carbon pricing trajectories actually align industrial investment incentives with climate targets.

Our central hypothesis is that carbon pricing alone, even at ambitious levels, will prove insufficient to achieve sectoral carbon budget compliance. Specifically, we expect that the Net Zero 2050 price trajectory—reaching \$150/tCO$_2$ by 2030 and \$450/tCO$_2$ by 2050—represents a necessary but still inadequate condition for staying within budget, due to capital stock inertia, infrastructure constraints, and technology cost barriers. Meanwhile, carbon pricing consistent with current policy trajectories (the NDCs scenario, reaching only \$40/tCO$_2$ by 2030 and \$100/tCO$_2$ by 2050) should systematically overshoot budget allocations, creating a dangerous policy-performance gap that undermines national climate commitments. If confirmed, this hypothesis would point to fundamental limitations in relying on carbon pricing as the primary instrument for industrial decarbonization.

Our results provide sobering confirmation of these concerns. Even under the most aggressive carbon price scenario—Net Zero 2050—POSCO's cumulative emissions still overshoot the carbon budget by 80 MtCO$_2$, or roughly 7\%. The Below 2$^\circ$C pathway produces a 603 MtCO$_2$ overshoot (+54\%), while the NDC scenario leads to an 871 MtCO$_2$ excess (+78\%). These overshoots occur despite the model's assumption of perfect foresight, rational optimization, and full technology availability—conditions far more favorable than real-world circumstances. The findings suggest that current carbon pricing trajectories are fundamentally inadequate for achieving climate targets in energy-intensive industries. Meeting those targets will require not just higher carbon prices, but also complementary policies that address infrastructure gaps, accelerate free allocation phase-outs, and reduce investment risks for low-carbon technologies. The remainder of this paper develops these arguments in detail and explores their implications for climate policy design.

