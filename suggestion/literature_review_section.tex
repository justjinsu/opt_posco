\section{Literature Review}

Understanding whether carbon pricing can drive industrial decarbonization requires synthesizing insights from three distinct research traditions that have largely developed in isolation. Engineering studies have mapped out technically feasible decarbonization pathways for steel production, identifying promising technologies but often treating policy instruments as exogenous parameters. Economic analyses have examined carbon pricing effectiveness in regulated industries, yet systematic evidence on heavy manufacturing remains surprisingly thin. Meanwhile, climate scientists have developed increasingly sophisticated carbon budget frameworks, but translating these global constraints into operational sectoral targets has proven elusive. The central gap—and the one this paper addresses—is that no existing research rigorously tests whether carbon pricing trajectories aligned with climate targets can actually drive the technology transitions required to meet sectoral carbon budgets. We begin by reviewing each literature stream in turn, highlighting both their contributions and limitations, before positioning our approach at their intersection.

\subsection{Steel sector decarbonization: Technology pathways and constraints}

Research on steel decarbonization has matured considerably over the past decade, evolving from single-technology assessments to system-level analyses that grapple with real-world constraints. The foundational work established a taxonomy of three main decarbonization routes \citep{IEA2020steel}: carbon management through retrofitting carbon capture and storage (CCUS) to existing blast furnace-basic oxygen furnace (BF-BOF) facilities; circularity and electrification via expanded use of scrap-based electric arc furnaces (EAF); and hydrogen-based direct reduction that replaces coking coal with H$_2$ as the primary reductant. This classification remains useful, though increasingly we recognize that real decarbonization strategies will likely combine elements from all three routes rather than betting on a single pathway.

Early techno-economic studies focused primarily on demonstrating technical feasibility and comparing levelized costs across technologies under static assumptions \citep{Vogl2018, Otto2017}. These analyses were valuable in establishing baseline cost estimates, but they treated technology costs as fixed and assumed perfect availability of necessary inputs—both problematic assumptions for long-term scenario analysis. More recent work has begun incorporating dynamic factors that matter critically for deployment. For instance, \citet{prammer2021steel} examined how technology learning curves might reduce hydrogen DRI costs over time, while \citet{ueckerdt2021potential} explored the co-evolution of hydrogen infrastructure and end-use applications. System-level constraints have also received growing attention: \citet{pauliuk2013global} documented global scrap availability limits that constrain how much steel production can shift to EAF routes, and \citet{wang2021hydrogen} mapped renewable hydrogen production potential, finding substantial regional variation in availability and cost. The most comprehensive synthesis came from \citet{MaterialEconomics2019}, whose European steel roadmap emphasized that decarbonization pathways depend critically on the timing and sequencing of policy interventions—get the timing wrong, and you risk either stranded assets or locked-in emissions.

Despite this progress, three important limitations remain. First, nearly all existing optimization models treat technology adoption as continuous decision variables that can be adjusted incrementally. This overlooks the fundamentally lumpy and irreversible nature of steel industry investment \citep{Griffin2020}. Blast furnaces are built as discrete units with 40-year lifespans; companies cannot simply adopt "30\% of a hydrogen DRI plant." This matters because it introduces path dependencies and option values that continuous models miss entirely. Second, while some studies examine specific regions, few integrate the full set of region-specific factors—policy frameworks, feedstock availability, infrastructure constraints, and industrial structure—that jointly shape feasible decarbonization pathways \citep{zhang2022steel}. A technology sequence that works for European steel may not transfer to Korea's context. Third, and most critically for our purposes, existing studies evaluate decarbonization scenarios against arbitrary emission targets or cost minimization objectives, but not against carbon budgets derived systematically from national climate commitments. Yet assessing whether policy instruments can deliver on actual climate targets is precisely what policymakers need to know.

\subsection{Carbon pricing effectiveness in energy-intensive industries}

If the steel engineering literature has made substantial progress, the same cannot be said for empirical evidence on carbon pricing effectiveness in heavy industry. The broader carbon pricing literature has documented responsiveness in sectors with relatively flexible production: power generation shows clear price sensitivity \citep{jarke2017carbon}, and oil refineries have adjusted product slates in response to EU ETS prices \citep{fowlie2016carbon}. But these sectors differ fundamentally from energy-intensive manufacturing, where production technologies are capital-intensive, long-lived, and offer limited short-run substitution possibilities.

For manufacturing specifically, the evidence remains limited and somewhat contradictory. \citet{calel2016innovation} found that EU ETS coverage accelerated low-carbon patenting in participating firms, suggesting that carbon pricing can stimulate innovation even if near-term abatement options are limited. However, \citet{martin2016industry} documented substantial emissions leakage, with regulated firms shifting production to unregulated facilities rather than reducing emissions intensity. This divergence hints at an important distinction: carbon pricing may change innovation incentives without necessarily changing near-term production and investment decisions, especially when free allocation shields firms from carbon costs.

The steel sector presents an even more challenging test case. \citet{sartor2012benchmark} showed that Phase III EU ETS free allocation—based on product-specific benchmarks—effectively removed carbon cost exposure for most steel producers, leaving them with little incentive to invest in new technologies despite rising carbon prices. Even when firms face carbon costs, price levels matter enormously. \citet{demailly2018european} estimated that carbon prices would need to exceed €50 per tonne CO$_2$ to make hydrogen-based DRI competitive with conventional blast furnaces at prevailing technology costs—a threshold the EU ETS only briefly reached during the 2008 financial crisis before collapsing. Whether the price levels now being contemplated for 2030 and beyond will prove sufficient remains an open question.

Research on emerging market carbon pricing systems has begun to fill in the picture, though the findings so far are not encouraging. \citet{kim2021kets} documented the first five years of Korea's emissions trading system, finding limited price discovery, persistent price suppression due to regulatory intervention, and minimal evidence of industrial restructuring. \citet{wang2021carbon} reported similar patterns in China's ETS pilots: prices remained low, volatility was high, and covered firms showed little change in investment behavior. These studies raise a troubling possibility: carbon pricing may be a necessary component of industrial climate policy, but current implementations often fail to generate the price signals needed to drive transformation.

What remains missing from this literature is any rigorous empirical test of whether planned carbon price trajectories—the ones actually embedded in national climate policies—can drive technology adoption at the scale and pace required for sectoral carbon budget compliance. Existing studies either examine historical behavior under low and uncertain carbon prices, or perform stylized analyses using arbitrary price assumptions disconnected from actual policy commitments. Neither approach answers the policy-relevant question: are current plans adequate?

\subsection{Sectoral carbon budgets and allocation mechanisms}

The concept of carbon budgets has become central to climate policy, yet its application to sectoral planning remains underdeveloped. The intellectual foundation came from climate science, particularly the finding that cumulative carbon emissions determine long-run temperature outcomes with remarkable precision \citep{matthews2009proportionality}. This relationship implies that limiting warming to any target requires staying within a finite "budget" of allowable emissions—a constraint that fundamentally changes how we think about climate policy.

Translating global budgets into actionable policy has proven challenging at multiple levels. At the global scale, \citet{rogelj2019new} refined estimates of remaining carbon budgets consistent with 1.5°C and 2°C temperature limits, accounting for recent emissions and updated climate sensitivities. But as \citet{millar2017emission} emphasized, meeting these budgets depends critically on near-term emission trajectories; delayed action rapidly erodes the remaining budget available for later decades. Moving from global to national budgets introduces thorny equity questions about how to allocate a finite resource across countries with vastly different historical responsibilities and development needs \citep{raupach2014sharing}. \citet{robiou2019national} proposed various allocation principles—equal per capita, grandfathering, capacity-based—each with different ethical foundations and practical implications.

Sectoral carbon budget allocation adds yet another layer of complexity \citep{gasser2018negative}. Should sectors with limited abatement options receive larger allocations? How should we account for differences in decarbonization costs, trade exposure, and social importance? \citet{kuramochi2018beyond} examined sectoral decomposition approaches, but struggled to identify clear principles for allocation that are both technically sound and politically acceptable. The practical result is that most national climate strategies remain vague about sectoral targets, offering aspirational goals rather than binding budgets.

For steel specifically, the stakes are particularly high. Given the sector's emissions intensity and the long lifetimes of production assets, \citet{bataille2018role} estimated that steel production could consume 15-20\% of the remaining global carbon budget under business-as-usual trajectories—far exceeding the sector's share of current GDP or employment. This raises urgent questions about how aggressive steel decarbonization must be, and whether available policy instruments can deliver the necessary transformation. \citet{Griffin2020} emphasized that sectoral carbon budgets should guide industrial policy design, yet acknowledged that we lack clear frameworks for deriving these budgets or assessing policy adequacy against them.

The fundamental gap is empirical: no existing research has tested whether current policy instruments—carbon pricing in particular—can actually deliver outcomes consistent with sectorally allocated carbon budgets. Without such tests, we cannot know whether planned policies are on track or require fundamental recalibration. This gap is especially concerning for energy-intensive industries like steel, where transformation timelines are long and getting policies wrong carries enormous consequences.

\subsection{Research contribution and positioning}

This study addresses the gaps identified above by integrating engineering-economic optimization with top-down carbon budget constraints to test carbon pricing adequacy in heavy industry. We make three main contributions.

First, we develop an integrated methodological framework that combines bottom-up technology optimization with carbon budget evaluation. Unlike previous steel optimization models that treat investments as continuous or use arbitrary emission targets, our mixed-integer formulation captures the discrete, lumpy nature of blast furnace campaigns while benchmarking results against carbon budgets derived systematically from national climate commitments. This allows us to quantify "policy-performance gaps"—the difference between what current policies are likely to deliver and what climate targets require.

Second, we provide what is, to our knowledge, the first quantitative assessment of whether carbon price trajectories aligned with major climate scenarios can drive technology transitions consistent with sectoral carbon budgets. Using Korea's steel sector as an empirically rich case study, we test three carbon price pathways from the Network for Greening the Financial System: Net Zero 2050, Below 2°C, and NDCs. This design directly addresses the question policymakers face: are our current plans sufficient, or do we need more aggressive policy intervention?

Third, we generate actionable policy insights by identifying specific carbon price thresholds that trigger different technology transitions and quantifying the magnitude of reforms needed to close emerging policy-performance gaps. Beyond simply documenting inadequacy, we provide guidance on what would be required for budget compliance—higher price floors, faster free allocation phase-outs, and complementary infrastructure policies.

Our approach bridges engineering-economic modeling with climate policy evaluation in a way that can extend to other energy-intensive sectors and jurisdictions. However, we should be clear about limitations. We focus on a single firm in one country, which limits generalizability though offers the advantage of institutional and technological specificity. We assume perfect foresight and cost-minimizing behavior, whereas real investment decisions involve uncertainty, organizational inertia, and multiple objectives beyond cost minimization. We do not model political economy constraints on carbon pricing—such as lobbying, public acceptance, or international competitiveness concerns—that often bind more tightly than technical or economic constraints. Finally, our discrete-time formulation (annual periods) cannot capture intra-year dynamics or the precise timing of investment campaigns. Despite these limitations, we believe the framework provides valuable insights into carbon pricing adequacy and establishes a template for policy evaluation that future research can refine and extend.
