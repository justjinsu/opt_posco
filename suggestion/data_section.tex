\section{Data and scenario design}

The model described above requires detailed parameterization across technology costs, commodity prices, emission factors, and demand projections. This section documents our data sources and calibration approach, providing the empirical foundation for our analysis. All monetary values are expressed in real 2024 USD unless otherwise specified.

\subsection{System boundary and company scope}

Our analysis focuses on POSCO's Scope 1 emissions from integrated steel production—the direct CO$_2$ emissions from iron ore reduction, steelmaking, and on-site energy combustion. This encompasses approximately 37.5 million tonnes per year of crude steel capacity across POSCO's major integrated facilities at Pohang and Gwangyang. We exclude upstream supply chain emissions (Scope 3) and downstream processing activities, focusing instead on the emissions covered by Korea's emissions trading system, where carbon pricing actually applies.

This boundary choice has important implications for interpreting our results. POSCO's asset base is overwhelmingly blast-furnace driven—the company mothballed its last carbon-steel EAF line in 2014 and now operates nine blast furnaces (five at Pohang, four at Gwangyang) that produce virtually all flat product output \citep{POSCO2023SR,KOSA2024Yearbook}. Residual EAF capacity of around 1.7 Mt/y exists in specialty subsidiaries producing stainless and tool steels, but these facilities serve niche markets unsuitable for replacing automotive-grade slab and plate production. Consequently, any expansion of scrap-based routes in our optimization must finance greenfield EAF modules and supporting scrap logistics infrastructure from scratch. This represents a structural hurdle largely absent in scrap-rich markets like the United States, where substantial EAF capacity already exists.

POSCO offers an ideal case study for several reasons beyond its sheer size. First, the company's market dominance—controlling approximately 60\% of Korean steel production—means that firm-level analysis generates direct insights for sectoral policy. Second, POSCO maintains technological diversity across integrated steelmaking, scrap processing capabilities, and emerging hydrogen-DRI development, making the technology portfolio optimization practically relevant rather than hypothetical. Third, comprehensive emissions reporting under K-ETS since 2015 provides robust calibration data unavailable for many steel producers globally. These factors make POSCO not just a large firm, but a particularly transparent and policy-relevant subject for analysis.

\subsection{Technology portfolio and route specification}

Our optimization model considers five technologically distinct production routes that span the full spectrum of steel decarbonization options. Each route is characterized by distinct input requirements, emission profiles, cost structures, and deployment constraints reflecting engineering and commercial realities.

\textbf{Route 1: Blast Furnace-Basic Oxygen Furnace (BF-BOF).} This is the incumbent technology, using coking coal to reduce iron ore in blast furnaces followed by oxygen steelmaking. BF-BOF represents the industry's workhorse technology with fully established supply chains and operational practices, but it also carries the highest emission intensity at 2.1 tCO$_2$ per tonne of steel. We model discrete blast furnace units with typical capacity of 4.0 Mt/y, reflecting the standard scale for modern integrated steelworks. Capital costs are set at \$800/tpy based on recent greenfield project benchmarks, with 40-year asset lifetimes capturing the long-term nature of these investments.

\textbf{Route 2: BF-BOF with Carbon Capture and Storage (BF-BOF+CCUS).} This route retrofits existing or new integrated steelworks with post-combustion CO$_2$ capture technology. CCUS captures approximately 80\% of process emissions, reducing the net emission factor from 2.1 to 0.42 tCO$_2$/t steel. We assume commercial availability from 2027 following POSCO's planned demonstration project completion at Pohang. The capital cost premium is substantial—65\% higher than conventional BF-BOF, or \$1,320/tpy—reflecting the additional equipment for CO$_2$ separation, compression, and transport infrastructure. Fixed operating costs also increase due to solvent makeup, compression energy, and maintenance requirements. This cost structure makes CCUS attractive only under sufficiently high carbon prices that justify the upfront investment.

\textbf{Route 3: Scrap-Electric Arc Furnace (Scrap-EAF).} Secondary steelmaking melts scrap steel using electric arc furnaces, bypassing the ore reduction step entirely. Emission intensity depends critically on electricity grid carbon content—under Korea's current grid mix, we estimate 0.15 tCO$_2$/t steel, which declines over time as the grid decarbonizes. EAF units are modeled with 2.0 Mt/y capacity reflecting typical mini-mill scale. While capital costs are lower than blast furnaces (\$600/tpy), scrap-EAF expansion faces two binding constraints in the Korean context: limited domestic scrap availability and quality requirements for premium flat products serving automotive and electronics markets. We impose an upper bound on scrap availability that grows from 8 Mt/y in 2025 to 12 Mt/y by 2050, reflecting gradual improvements in scrap collection and quality sorting infrastructure.

\textbf{Route 4: Natural Gas Direct Reduced Iron-EAF (NG-DRI-EAF).} This route produces iron using natural gas reduction followed by electric remelting, serving as a bridge technology between conventional and fully decarbonized pathways. Emission intensity of 0.8 tCO$_2$/t steel is considerably lower than BF-BOF but higher than hydrogen-based alternatives. NG-DRI technology is commercially mature with no timing constraints, making it available throughout the optimization horizon. Capital costs of \$1,000/tpy reflect the integrated DRI-EAF system. While this route avoids coking coal entirely, it remains fossil fuel-dependent and thus represents at best a transitional solution.

\textbf{Route 5: Hydrogen Direct Reduced Iron-EAF (H$_2$-DRI-EAF).} Primary steelmaking using hydrogen as the reductant offers near-zero direct emissions—we estimate 0.2 tCO$_2$/t steel accounting for residual natural gas use and auxiliary processes. This includes POSCO's proprietary HyREX technology, which the company is developing for commercial deployment by 2030. Capital costs are the highest of any route at \$1,500/tpy, reflecting both the nascent state of the technology and substantial hydrogen infrastructure requirements (storage, pipeline connections, backup systems). The key operational challenge is hydrogen cost: at our baseline trajectory declining from \$4.50/kg in 2030 to \$2.80/kg by 2050, hydrogen represents a substantial share of variable costs. We model hydrogen availability as unconstrained within the planning horizon, a strong assumption that we relax in sensitivity analysis. Process intensity is set at 45 kg H$_2$ per tonne steel based on POSCO's pilot plant specifications.

This five-route framework captures the essential trade-offs facing steel decarbonization: incumbent technologies are cheap but carbon-intensive; carbon capture can retrofit existing assets but at high cost; scrap-based routes are low-carbon but supply-constrained; and hydrogen routes promise deep decarbonization but face high costs and infrastructure challenges. By letting the model choose among these options, we can trace out how different carbon price levels shift the balance among pathways.

\subsection{Carbon price scenarios and policy framework}

Our three carbon price trajectories come from the Network for Greening the Financial System (NGFS) Phase 5 scenarios \citep{NGFS2024}, which provide globally consistent climate policy pathways widely used by central banks and financial regulators for climate risk assessment. We focus on the Asia-Pacific advanced economy grouping, which most closely matches Korea's income level and industrial structure.

\textbf{Net Zero 2050 (NZ2050):} This aggressive scenario is consistent with limiting warming to 1.5°C and achieving global net-zero CO$_2$ emissions by 2050. Carbon prices climb rapidly from \$50/tCO$_2$ in 2025 to \$150/tCO$_2$ in 2030, reaching \$450/tCO$_2$ by 2050. The trajectory reflects the NGFS assumption that delayed action necessitates steeper price increases in later periods to meet cumulative emission constraints. This scenario represents the high end of plausible policy ambition given current climate commitments.

\textbf{Below 2°C (B2C):} This moderate ambition pathway aligns with limiting warming to well below 2°C, as specified in the Paris Agreement. Carbon prices reach \$30/tCO$_2$ in 2025, \$80/tCO$_2$ in 2030, and \$240/tCO$_2$ in 2050. While still ambitious by current standards, this trajectory reflects less aggressive near-term action and more gradual policy tightening, potentially allowing for more incremental technology transitions.

\textbf{Nationally Determined Contributions (NDCs):} This scenario assumes countries implement their current climate pledges but do not strengthen them significantly beyond stated commitments. Carbon prices remain relatively modest at \$15/tCO$_2$ in 2025, rise to \$40/tCO$_2$ by 2030, and plateau at \$100/tCO$_2$ in 2050. This represents roughly a continuation of current K-ETS price dynamics with gradual strengthening, but without the dramatic increases assumed in more ambitious scenarios.

Free allocation under K-ETS follows Korea's stated industrial decarbonization timeline. We model allocation declining from 8.5 MtCO$_2$ in 2025 to 4.2 MtCO$_2$ in 2030, consistent with the 40\% NDC reduction target for the industrial sector. Post-2030, allocation phases out approximately linearly, reaching 1.0 MtCO$_2$ by 2050 to ensure meaningful carbon price exposure during the critical 2030-2050 technology transition period. This phase-out schedule is more aggressive than current policy but aligned with Korea's carbon neutrality aspirations.

The choice of NGFS scenarios is deliberate. Unlike arbitrary carbon price assumptions, these scenarios are grounded in integrated assessment models that link emissions, climate outcomes, and economic activity. They provide internally consistent policy pathways that connect near-term prices to long-term temperature targets, making them more realistic than ad hoc price trajectories. Moreover, their use by financial institutions for climate stress testing means our results speak directly to stakeholders assessing transition risks in steel sector investments.

\subsection{Demand projection and market context}

POSCO's steel demand trajectory reflects Korea's economic development stage and material intensity evolution. We project demand growing modestly from 37.5 Mt in 2025 to peak at 39.2 Mt in 2035, before declining gradually to 35.8 Mt by 2050. This inverted-U profile is characteristic of developed economies transitioning from material-intensive growth phases toward service-dominated economies with improved material efficiency.

The underlying dynamics driving this trajectory include continued infrastructure investment through the mid-2030s (roads, bridges, urban development), sustained automotive and shipbuilding sector demand that matures gradually, and post-2035 material efficiency improvements through circular economy practices, light-weighting, and longer product lifespans. These projections align with Korea Development Bank steel sector forecasts and World Steel Association long-term demand scenarios for developed Asian economies.

We assume POSCO maintains constant market share throughout the analysis period, abstracting from competitive dynamics to focus purely on technology transition incentives under carbon pricing. This assumption is defensible given POSCO's dominant market position and the 25-year timeframe spanning multiple investment cycles. However, we acknowledge this overlooks potential market share shifts due to imports, domestic competition, or demand destruction from high carbon costs—factors we discuss in our limitations section.

\subsection{Input cost parameterization and price projections}

Commodity price trajectories combine World Bank Commodity Market Outlook projections with Korea-specific adjustments for transport costs, quality premiums, and local supply-demand dynamics. We detail the major inputs:

\textbf{Iron ore:} FOB Australia prices for 62\% Fe fines declining gradually from \$95/t (2025) to \$85/t (2050), reflecting anticipated supply expansion in Brazil and West Africa alongside demand maturation in China. Korea-delivered prices add \$15/t for freight and port handling, yielding a delivered cost of \$110/t declining to \$100/t.

\textbf{Coking coal:} Premium hard coking coal (required for blast furnace operation) priced at \$180/t (2025) declining to \$160/t (2050). The decline reflects both substitution toward lower-carbon production methods globally and potential carbon border adjustments that raise relative costs of coal-intensive production.

\textbf{Scrap steel:} Domestic heavy melting scrap prices rising from \$420/t (2025) to \$480/t (2050). The increase reflects growing competition for scrap as EAF capacity expands globally, quality premiums for automotive-grade scrap, and improvements in collection infrastructure. Scrap prices are deliberately modeled higher than international benchmarks to reflect Korea's structural scrap deficit and import dependence.

\textbf{Natural gas:} Korean LNG import prices declining from \$12/GJ (2025) to \$10/GJ (2050), following global supply expansion from US, Australia, and Qatar alongside infrastructure development that reduces transport costs. This trajectory is conservative relative to some forecasts anticipating steeper declines.

\textbf{Electricity:} Industrial tariffs stable at \$75/MWh in real terms throughout the analysis period, reflecting KEPCO's regulated pricing structure under gradual grid decarbonization. We conduct sensitivity analysis with ±30\% price variation to capture uncertainty around future electricity costs and potential subsidy reforms.

\textbf{Hydrogen:} Our baseline trajectory assumes costs declining from \$4.50/kg when commercial production begins in 2030 to \$2.80/kg by 2050, reflecting anticipated reductions in electrolyzer capital costs and renewable electricity prices. This baseline draws from IEA hydrogen projections and Korean government hydrogen roadmap targets. We also test an optimistic scenario with costs 20\% lower, representing accelerated technology learning and stronger policy support. Hydrogen costs remain the single largest source of uncertainty in our analysis.

Process intensities reflect best-available technology performance: BF-BOF requires 1.5 t iron ore and 0.5 t coking coal per tonne steel; Scrap-EAF consumes 1.05 t scrap and 0.5 MWh electricity per tonne; H$_2$-DRI utilizes 45 kg hydrogen per tonne based on POSCO HyREX pilot plant data. These intensities incorporate realistic yield losses, energy requirements, and auxiliary material consumption, making them more conservative than theoretical stoichiometric requirements.

\subsection{Technology costs and investment parameters}

Capital expenditure estimates reflect greenfield investment costs for new capacity, calibrated using industry benchmarks, engineering studies, and POSCO's own project disclosures where available. Table~\ref{tab:tech-costs} summarizes unit module sizes, CAPEX, and fixed operating costs for each technology route. Table~\ref{tab:tech-intensity} reports the associated Scope 1 emission factors and feedstock intensities.

Several parameters merit specific discussion. Capital costs differ markedly across technologies—from \$600/tpy for scrap-EAF to \$1,500/tpy for hydrogen DRI—reflecting both technological maturity and infrastructure requirements. These estimates draw on engineering studies compiled by Material Economics, IEA technology roadmaps, and Korean steel industry assessments \citep{MaterialEconomics2019,kuramochi2018beyond,prammer2021steel}. We note that hydrogen DRI costs carry particularly high uncertainty given limited commercial deployment; our sensitivity analysis explores this uncertainty systematically.

Fixed operating costs range from \$80 to \$250 per tonne capacity per year, capturing maintenance, labor, utilities, and overhead expenses that must be paid regardless of production levels. CCUS and hydrogen routes face higher fixed costs due to specialized equipment maintenance, solvent makeup, compression energy, and buffer capacity requirements. These cost differentials are critical for understanding technology adoption patterns—high fixed costs favor baseload operation while low fixed costs enable flexible operation.

CCUS capture efficiency is modeled at 80\% based on current post-combustion amine technology, with gross emissions identical to conventional BF-BOF. This means CCUS reduces net emissions from 2.1 to 0.42 tCO$_2$/t steel. We stress-test this assumption in sensitivity analysis by examining cases with lower capture rates (70\%) and higher energy penalties. The key limitation is our assumption that CO$_2$ transport and storage infrastructure exists at reasonable cost—a strong assumption for Korea that we discuss further in our policy implications.

All costs are expressed in real 2024 terms with a 5\% discount rate applied consistently across scenarios. This rate reflects typical corporate hurdle rates for long-term industrial projects and is standard in IEA energy system analyses. We test sensitivity to this assumption with alternative rates of 3\% (social discounting) and 7\% (high private discount rate).

\begin{table}[ht]
  \centering
  \caption{Technology cost and capacity assumptions}
  \label{tab:tech-costs}
  \begin{threeparttable}
  \begin{tabular}{@{}lccc@{}}
    \toprule
    Route & Unit capacity (Mt/y) & CAPEX (USD/tpy) & Fixed OPEX (USD/tpy) \\
    \midrule
    BF--BOF & 5.0 & 1{,}000 & 100 \\
    BF--BOF+CCUS & 5.0 & 1{,}400 & 150 \\
    FINEX--BOF & 3.0 & 1{,}200 & 120 \\
    Scrap--EAF & 2.0 & 800 & 80 \\
    NG--DRI--EAF & 2.5 & 1{,}800 & 180 \\
    H$_2$--DRI--EAF & 2.0 & 2{,}500 & 200 \\
    HyREX & 1.5 & 3{,}000 & 250 \\
    \bottomrule
  \end{tabular}
  \begin{tablenotes}
    \footnotesize
    \item Notes: Parameters sourced from the engineering data pack (\texttt{data/posco\_parameters\_consolidated.xlsx}; see \citet{MaterialEconomics2019,prammer2021steel,kuramochi2018beyond} for benchmark values). Unit capacities correspond to discrete module sizes in the optimisation. CAPEX/OPEX in real 2024 USD. Scope~1 emission factors and feedstock intensities are reported in Table~\ref{tab:tech-intensity}.
  \end{tablenotes}
  \end{threeparttable}
\end{table}

\begin{table}[ht]
  \centering
  \caption{Scope-1 emission factors and feedstock intensities by technology route}
  \label{tab:tech-intensity}
  \begin{threeparttable}
  \begin{tabular}{@{}lcccccc@{}}
    \toprule
    \multirow{2}{*}{Route} & Scope-1 EF & \multicolumn{5}{c}{Feedstock and energy intensity (per t crude steel)} \\
    \cmidrule(lr){3-7}
     & (tCO$_2$/t) & Iron ore (t) & Scrap (t) & Natural gas (GJ) & Electricity (MWh) & H$_2$ (kg) \\
    \midrule
    BF--BOF & 2.10 & 1.50 & 0.10 & 0.50 & 0.60 & 0 \\
    BF--BOF+CCUS & 2.10\tnote{a} & 1.50 & 0.10 & 0.50 & 0.80 & 0 \\
    FINEX--BOF & 1.95 & 1.40 & 0.15 & 0.40 & 0.65 & 0 \\
    Scrap--EAF & 0.45 & 0.05 & 1.05 & 1.80 & 0.50 & 0 \\
    NG--DRI--EAF & 1.20 & 1.30 & 0.20 & 12.00 & 0.70 & 0 \\
    H$_2$--DRI--EAF & 0.15 & 1.25 & 0.15 & 0.00 & 2.20 & 45 \\
    HyREX & 0.10 & 1.20 & 0.10 & 0.00 & 2.50 & 50 \\
    \bottomrule
  \end{tabular}
  \begin{tablenotes}
    \footnotesize
    \item Notes: Intensities drawn from the \texttt{process\_intensity} and \texttt{ef\_scope1} sheets in \texttt{data/posco\_parameters\_consolidated.xlsx}. Figures reflect steady-state operation with 90\% utilisation.
    \item[a] CCUS routes assume gross emissions of 2.10~tCO$_2$/t with 80\% capture efficiency; the net factor used in the model is 0.42~tCO$_2$/t.
  \end{tablenotes}
  \end{threeparttable}
\end{table}

