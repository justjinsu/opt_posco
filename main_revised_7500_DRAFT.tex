\documentclass[review,12pt]{elsarticle}
\usepackage{lineno,hyperref,graphicx,amsmath,booktabs,enumitem}
\modulolinenumbers[5]
\journal{Energy Policy}

\bibliographystyle{elsarticle-harv}

\begin{document}

\begin{frontmatter}

\title{Can carbon pricing justify Korea's hydrogen steel transition? Testing POSCO's decarbonization strategy against sectoral carbon budgets}

\author[planit]{Jinsu Park}
\address[planit]{PLANiT Institute, Seoul, Republic of Korea}

\section*{Highlights}
\begin{itemize}[leftmargin=*]
  \item \textbf{Hydrogen pathway viability}: Under Net Zero carbon pricing (\$383/tCO$_2$ by 2030, \$638/tCO$_2$ by 2050), hydrogen-based DRI captures 41\% of POSCO's 2050 output with cumulative emissions of 1,169~MtCO$_2$—overshooting the sectoral budget by only 5.3\%, validating POSCO's hydrogen strategy.
  \item \textbf{Binary pricing outcome}: Ambitious Net Zero pricing enables hydrogen deployment and near-budget alignment, while moderate scenarios (Below~2$^\circ$C, NDCs) fail catastrophically with +78\% overshoots, demonstrating that half-measures cannot justify transformative hydrogen investment.
  \item \textbf{Infrastructure dependency}: Hydrogen pathway requires dedicated supply chains (production, transport, pellet feedstock) but proves cost-competitive at \$815/t versus \$861/t under business-as-usual, showing early capital investment substitutes for recurring carbon costs.
  \item \textbf{Price thresholds matter}: Hydrogen DRI becomes economic when carbon prices exceed \$350--400/tCO$_2$, explaining why only Net Zero pricing justifies POSCO's hydrogen investment while weaker scenarios lock in conventional blast furnaces.
  \item \textbf{Policy enablers}: Realising the hydrogen pathway requires sustained price floors above \$600/tCO$_2$ by 2050, faster free-allocation phase-out, government co-investment in H$_2$ infrastructure, and long-term supply contracts that de-risk hydrogen steelmaking.
\end{itemize}

\begin{abstract}
Can carbon pricing justify Korea's ambitious shift toward hydrogen steelmaking? We examine this question using a mixed-integer optimisation model that evaluates POSCO's technology choices under NGFS Phase V carbon-price trajectories (Net Zero 2050: \$383/tCO$_2$ in 2030, \$638/tCO$_2$ in 2050; Below~2$^\circ$C: \$71/\$166; NDCs: \$118/\$130) against a 1,110~MtCO$_2$ sectoral budget for 2025--2050. We find that ambitious Net Zero pricing enables hydrogen-based direct reduction to capture 41\% of 2050 output, achieving cumulative emissions of 1,169~MtCO$_2$—just 5.3\% above budget. This validates POSCO's hydrogen strategy: the optimisation independently selects H$_2$-DRI as the primary decarbonisation route when carbon prices sustain above \$350--400/tCO$_2$. Moderate pricing scenarios fail catastrophically—Below~2$^\circ$C and NDCs both overshoot by 78\%—because prices never justify the upfront hydrogen investment, locking in conventional blast furnaces through 2050. The hydrogen pathway proves cost-competitive at \$815/t steel versus \$861/t under business-as-usual, as early capital substitutes for recurring carbon costs. However, realising this pathway requires more than pricing alone: government must co-invest in hydrogen infrastructure (production, transport, pellet feedstock), accelerate free-allocation phase-out so firms face real carbon costs, and provide long-term supply contracts that de-risk the transition. Our results show carbon pricing can work—but only when sustained at levels that justify transformative investment in hydrogen technology.
\end{abstract}

\begin{keyword}
Hydrogen steelmaking \sep carbon pricing \sep steel decarbonization \sep mixed-integer optimization \sep Korea ETS \sep NGFS scenarios \sep industrial policy
\JEL{Q41, Q54, L61}
\end{keyword}
\end{frontmatter}

\section{Introduction}

Can carbon pricing justify large-scale hydrogen investment in steelmaking? This question matters urgently for Korea, where POSCO—the world's sixth-largest steel producer—has committed to hydrogen-based direct reduction as its primary decarbonization pathway. The company's HyREX demonstration plant signals this strategic bet, but whether carbon market signals can make such investments profitable remains untested.

The stakes are considerable. POSCO's emissions alone account for 10\% of Korea's greenhouse gas inventory, roughly equivalent to Belgium's entire carbon footprint. Moving from today's coal-intensive blast furnaces to hydrogen steelmaking requires wholesale infrastructure replacement—individual plants cost over \$2 billion and operate for 25-40 years. These aren't incremental adjustments but irreversible commitments that lock in technology pathways for decades.

Korea's emissions trading system provides the policy backdrop. Launched in 2015, the K-ETS now covers 70\% of national emissions. Yet POSCO historically received free allowances covering 95\% of its emissions, effectively shielding the sector from carbon costs. This protection is scheduled to decline as Korea pursues its 2050 carbon neutrality goal, gradually exposing steel producers to meaningful price signals. Whether these evolving prices can trigger the hydrogen transition remains an open question.

We examine this using mixed-integer optimization of POSCO's technology portfolio under three NGFS Phase V carbon price scenarios: Net Zero 2050 (\$383/tCO₂ by 2030, \$638 by 2050), Below 2°C (\$71/\$166), and NDCs (\$118/\$130). The model minimizes total system costs—capital, operations, and carbon compliance—while respecting technology constraints, feedstock limits, and Korea's sectoral carbon budget of 1,110 MtCO₂ for 2025-2050.

Our central finding validates POSCO's hydrogen strategy. Under Net Zero pricing, hydrogen DRI captures 41\% of 2050 output, achieving cumulative emissions of 1,169 MtCO₂—just 5.3\% above budget. Moderate pricing scenarios fail catastrophically, with both Below 2°C and NDCs overshooting by 78\%, as prices never justify upfront hydrogen investment. The hydrogen pathway proves cost-competitive (\$815/t versus \$861/t under business-as-usual), but realizing it requires more than pricing alone: government must co-invest in infrastructure, accelerate free-allocation phase-out, and provide long-term contracts that de-risk the transition.

